\chapter{Introduction}
\label{ch:intro:intro}

For humans and other animals, in any context, adaptive behavior is defined as executing an action that would maximize immediate or future rewards, while minimizing energy expenditure.
For example, consider any solitary hunter that would wait for the right time to attack:
    when the prey is most distracted or vulnerable.
However, in some situations, the appearance of a sensory cue will not only indicate which action should be performed, but also how long, after the appearance of the cue, this action must be initiated~\cite{Balsam2009Map, Nobre2018NatRevNeurosci}.
For instance, athletes performing sprint races learn by experience that the \textit{go} command, signaling them to start running, will be given in two seconds after the \textit{set} command.
False starts, i.e., beginning to run just before the \textit{go} command, demonstrate that athletes estimated the~2~second interval between the \textit{set} and \textit{go} commands.
More generally, the ability of animals to exploit temporal regularities in nature is crucial for survival:
    the appearance of a sensory cue at a given time can predict food availability, predator attack, or mating opportunity~\cite{Kacelnik2002,Gallistel1990book}.
\par
In this chapter, first I introduce the taxonomy used in the timing literature to establish a reference for this manuscript.
Then I discuss two possible mechanisms that could give rise to the perception of elapsed time.
Next, I will review some evidence of how either mechanism is thought to be implemented in the brain, focusing on the role of a certain brain structure of interest.
Finally, I will lay out the question and the hypothesis underlying this work and the structure of the following chapters.

\section[Time Taxonomy]{Time Taxonomy} \label{ch:intro:taxonomy}
It is important to point out different categories of tasks used to study timing.\footnotemark
Appropriate classification of a phenomenon, alone, could lead to scientific advances.
\footnotetext{
    This section follows the arguments presented by~\citeauthor{Paton2018NeuronRev} in~\cite{Paton2018NeuronRev}.
    }
First step toward a taxonomy of time is to define what could be considered a timing task.
Not every task with a temporal dependency is regarded as a time estimation task.
A timing task requires an explicit understanding of a given duration, i.e., one would need a clock to solve the task.
For example, judging which of any two sensory stimuli occurred first does not require a timing device to solve and hence, is not a timing task.
On the other hand, judging which of those stimuli were longer, indeed is a timing task, since it cannot be solved without any reference for time.
\par
It is not perfectly clear, but there is some consensus over principal dimensions of the taxonomy of time.

\paragraph{Subsecond vs.\ Suprasecond Timing.} \label{ch:intro:taxonomy:SUBvsSUPRA}
There is ample evidence that timing relies on different mechanisms for short and long timescales~\cite[see][]{Paton2018NeuronRev}.
Although the boundary is not definite, for timescales relevant to this work, short intervals are several tens of milliseconds ($50-100$~ms), and long intervals include several hundred milliseconds to several seconds.
% Interestingly, subsecond intervals are closer to natural rhythmicities of the body, e.g., respiration and heart beat.

\paragraph{Interval vs.\ Pattern Timing.} \label{ch:intro:taxonomy:INTvsPAT}
There is evidence of differential neural mechanisms at play for simple timing tasks (such as reproducing a duration) as  opposed to tasks where the global temporal structure of the stimuli is determinant (such as recognizing the tempo of a song)~\cite{teki2011}.

\paragraph{Sensory vs.\ Motor Timing.} \label{ch:intro:taxonomy:SENvsMOT}
This dimension of time taxonomy, not unlike the other two, is a continuum.
In sensory tasks the subject analyzes the temporal information in the external world and reports their decision, such as an interval discrimination task.
Motor timing tasks, on the other hand, require a timely motor response, with no sensory cue --- such as delayed blinking in response to a conditioned stimulus.
While some tasks can be considered exclusively motor, or sensory, most tasks possess both sensory and motor components, namely, reproducing a temporal pattern, e.g., a Morse code.
\section{Internal Time Estimation}
\label{ch:intro:InternalTimeEstimation}

Understanding how animals adapt their behavior to time intervals of various durations is challenging, because unlike sensory modalities (vision, olfaction, audition), time is not a material entity and animals are not equipped with a sense organ for time perception.
Time perception in the timescale of a few seconds (compared to, say, 100~ms) seems to be even more puzzling, since it is much longer than intrinsic properties of neural function~\cite{Buhusi2005NatRevNeuro}. 
\par
One influential idea in the field of systems neuroscience, hereafter referred to as the \textit{internal clock}, posits that complex nervous systems have acquired the ability to estimate elapsed time and use this representation to estimate if the duration of a given time interval is similar to (or different from) a previously learned interval~\cite{Gibbon1977, Miali1989NeuComp, Gouvea2015Elife,Wittmann2013NatRevNeurosci}.
Irrespective of its exact neural implementation (which will be discussed later), the internal clock works according to the following principle.
Once a cue appears to signal the beginning of a time interval (e.g., the \textit{set} command in sprint races), the neuronal time quanta begin to accumulate.
Time interval estimation consists in comparing the magnitude of this ongoing accumulation with a stored value determined through experience (e.g., multiple exposures to time intervals between \textit{set} and \textit{go} commands in sprint races)~\cite{Simen2011JNeuro}.
Errors in counting neuronal time quanta will accumulate with time too.
Consequently, such a mechanism predicts that time estimation accuracy should degrade proportionally to the duration of the time interval, which has been verified in humans and animals~\cite{Gibbon1977, Lejeune1991LearnMotiv, Rakitin1998, Whitaker2003JExpPsy, Zarco2009JNeurophys, Hinton2004, JazayeriNN2018, Gallistel2004}.
This is a feature of timing, generally referred to as the \emph{scalar property}, and resembles the Weber's law in sensory perception.\footnotemark
\footnotetext{
    ``Formulated by Ernst Weber in 1831 to explain the relationship between the physical intensity of a stimulus and the sensory experience that it causes.
    Weber’s Law states that the increase in a stimulus needed to produce a just-noticeable difference is constant.
    Later, Gustav Fechner (1801-1887) generalized Weber's law by proposing that sensation increases as the logarithm of stimulus intensity: $S = k logI$, where $S=$ subjective experience, $I=$ physical intensity, and $k =$ constant."~\cite{Buhusi2005NatRevNeuro}
    }
% \par
% A variety of neuronal correlates of time have been reported (duration-selective neurons~\cite{DUYSENS1996}, ramping firing rate~\cite{Emmons2017JNeuro}, neuronal population~\cite{Merchant2011PNAS}, network state~\cite{Bakhurin2017JNeuro}, \ldots).
% These correlates are expressed by either individual neurons, or neuronal ensembles of various sizes recorded in multiple brain regions (e.g., prefrontal cortex~\cite{wiener2010image}, sensory cortices~\cite{Zhou2010}, cerebellum~\cite{johansson2014PNAS}, and \acrlong{bg}~\cite{Gouvea2015Elife}).
% Thus, it is unlikely that a single internal clock could be isolated from the multitude of temporally-ordered neuronal activities in the brain~\cite{Paton2018NeuronRev}.
% Moreover, how animals could know which internal clock they should watch remains speculative.
\par
% Temporal representations are relevant to various behavioral conditions, engaging several brain functions (sensory perception, memory, motor control, attention)~\cite{Buzsaki2017SciRev}, and they are intrinsic to the activity of ensembles of neurons in the brain (i.e., neuronal activity covaries with time)~\cite{MoserNature2018}.
Another proposal suggests that time estimation ultimately relies on task-specific emergent properties of interacting neuronal networks, rather than a pure time-dedicated internal clock~\cite{Paton2018NeuronRev}.
Such an \emph{emergent} clock also has the assumption that the origin of time perception is internal, i.e, organisms infer the elapsed time purely from their neuronal dynamics.

\subsection{Central Clock}
\label{ch:intro:InternalTimeEstimation:Central}

It has been long proposed that a central clock provides temporal information for organisms~\cite{gibbon1984AnnalsNYAS, Killeen1988}.
The \emph{pacemaker-accumulator} model is the most prominent computational account of such a central clock.
In essence, this model postulates:
    a pacemaker, which generates periodic pulses at intervals shorter than those being estimated;
    a switch, that following training, gates pulses through for a certain duration;
    an accumulator, downstream of the switch, that records the number of pulses in working memory;
    a reference memory, that holds the number of pulses that previously have been reinforced;
    a comparator, which determines whether the accumulated value is close enough to the reference value to warrant a response or not~\cite{gibbon1984AnnalsNYAS}.
This model explains the scalar property by introducing sources of variability to its components.
The pacemaker-accumulator model has many advantages:
    it is very straightforward, intuitive, and biologically feasible;
    it has clear separation of memory and decision-making systems, which could map to neural structures;
    and it is extremely successful in predicting behavioral data, given its simplicity~\cite{Buhusi2005NatRevNeuro}.
\par
Other internally-driven models of temporal processing have been proposed as well.
The \emph{Beat-frequency model} is another dedicated model for interval timing~\cite{Paton2018NeuronRev}.
In this model, different intervals are decoded from a bank of oscillators with different frequencies.
Subgroups of such oscillators may be in the same phase at intervals much greater than those of individual oscillators.
For example, three oscillators with periods of~5,~8, and~11~s are in the same phase every 440~s.\footnotemark
\footnotetext{
    Mathematically, for any number of oscillators, it will be their \textit{least common multiple}.
    }
Hence, by choosing various subgroups and detecting the time at which they are phase-locked, one could generate a wide range of intervals.
This model is also biologically feasible, as each oscillator could be as simple as a single neuron with a constant firing rate.
Consider a series of these \textit{oscillatory} neurons being reset with the stimulus (at the beginning of the interval).
At any point in time, a their spiking could be observed by a downstream structure.
A subset of these neurons that fire at the time of reinforcement (i.e., the end of the interval) could represent a neural code for this particular interval~\cite{Matell2004CogBrainRes}.
Similarly, other subsets could encode different intervals.
Among others, \Citeauthor{Miali1989NeuComp} simulated the beat-frequency model with 500 units oscillating at~5 to~15~Hz~\cite{Miali1989NeuComp}.
One output unit received single synapses from every unit and the strength of each synapse followed a simplified Hebbian rule.
This model managed to learn to encode intervals ranging from~200~ms to~10~s.
\par
Finally, there is another class of models based on ramping activity of neurons.
These models propose that a linear metric of elapsed time is encoded in decreasing/increasing firing rate of neurons~\cite{Paton2018NeuronRev}.
Crucially, the slope of the ramping must correlate negatively with the duration of the interval, since the peak firing rate is relatively constant~\cite{Jazayeri2015CurrBiol}.
Moreover, neurons have timescales of tens of milliseconds, thus, for these model to account for time estimation in behaviorally-relevant timescales, i,e., several hundreds of milliseconds to seconds, there must be a feedback mechanism.
Simulations by \citeauthor{Gavornik20009PNAS} demonstrate that recurrent excitatory synapses could provide such a feedback signal~\cite{Gavornik20009PNAS}.
In this network, activity of each neuron, if isolated, would decay after stimulus presentation.
However, by introducing recurrent connections, lateral propagation of activity in the network decreases each neuron's activity decay rate in response to a stimulus.
In other words, the network modifies the temporal properties of the response of individual neurons, which could translate to elapsed time representation.

\subsection{Emergent Clock}
\label{ch:intro:InternalTimeEstimation:Emergent}

A different class of models postulate that representations of time emerge from distributed dynamics of neural networks.
These models differ from those discussed in \autoref{ch:intro:InternalTimeEstimation:Central} in that these models are not localized, i.e., they involve different brain areas, however, they similarly assume time estimation is internally driven.
These models assume that sensory, motor, and cognitive processes that are not specifically dedicated to timing might form networks that (after training) act as interval timers~\cite{Wittmann2013NatRevNeurosci}.
\par
One type of such models, namely state-dependent networks, proposes that neural networks inherently contain temporal information as a result of their complexity.
In a seminal work, \citeauthor{Karmarkar2007Neuron} simulated a network of 400 excitatory and 100 inhibitory neurons, recurrently connected and exhibiting synaptic plasticity~\cite{Karmarkar2007Neuron}.
This network was then exposed to two identical events, 100~ms apart (e.g., two auditory tones).
Due to complex synaptic processes, the state of the network at any point in time after the presentation of the first stimulus would be different.
Thus, the population response to the second stimulus inherently encodes the duration between the two stimuli.
In this fashion, various intervals could be decoded from dynamics of ever more complex networks.
Indeed, in a more recent work, \citeauthor{Perez2018JNeurosci} simulated a recurrent network of 800 excitatory and 200 inhibitory neurons~\cite{Perez2018JNeurosci}.
The neurons were randomly connected and received two membrane currents induced by the input (one inhibitory, one excitatory).
In addition, each neuron in the network also received two recurrent inputs.
All of the synaptic currents followed time-varying dynamics.
These temporal synaptic properties (such as time constant of neurotransmitter release, inhibitory input current dynamics,~\ldots) allowed an optimal Bayesian decoder to produce interval-selective responses, in the range of several hundred milliseconds.
This network, given parameter values within physiological range, could demonstrate scalar property as well.
\par
This, by no means, is a comprehensive review of all the literature on timing models and that is not the focus of this manuscript.
There are numerous articles proposing different neurocomputational models (using ramping activity, drift diffusion, synfire chain, coincidence detector,~\ldots) to account for psychophysical evidence of timing behavior in humans and other animals.
\Citeauthor{Paton2018NeuronRev} review many of these models in a recent paper~\cite{Paton2018NeuronRev}.


\subsection[Time Taxonomy]{Time Taxonomy} \label{ch:intro:taxonomy}
Appropriate classification of a phenomenon, alone, could lead to scientific advances.\footnotemark
\footnotetext{
    This section follows the arguments presented by~\citeauthor{Paton2018NeuronRev} in~\cite{Paton2018NeuronRev}.
    }
First step toward a taxonomy of time is to define what could be considered a timing task.
Not every task with a temporal dependency is regarded as a time estimation task.
A timing task requires an explicit understanding of a given duration, i.e., one would need a clock to solve the task.
For example, judging which of any two sensory stimuli occurred first does not require a timing device to solve and hence, is not a timing task.
On the other hand, judging which of those stimuli were longer, indeed is a timing task, since it cannot be solved without any reference for time.
\par
It is not perfectly clear, but there is some consensus over principal dimensions of the taxonomy of time.

\paragraph{Subsecond vs. Suprasecond Timing.} \label{ch:intro:taxonomy:SUBvsSUPRA}
There is ample evidence that timing relies on different mechanisms for short and long timescales~\cite[see][]{Paton2018NeuronRev}.
Although the boundary is not definite, for timescales relevant to this work, short intervals are several tens of milliseconds ($50-100$~ms), and long intervals include several hundred milliseconds to several seconds.
% Interestingly, subsecond intervals are closer to natural rhythmicities of the body, e.g., respiration and heart beat.

\paragraph{Interval vs. Pattern Timing.} \label{ch:intro:taxonomy:INTvsPAT}
There is evidence of differential neural mechanisms at play for simple timing tasks (such as reproducing a duration) as  opposed to tasks where the global temporal structure of the stimuli is determinant (such as recognizing the tempo of a song)~\cite{teki2011}.

\paragraph{Sensory vs. Motor Timing.} \label{ch:intro:taxonomy:SENvsMOT}
This dimension of time taxonomy, not unlike the other two, is a continuum.
In sensory tasks the subject analyzes the temporal information in the external world and reports their decision, such as an interval discrimination task.
Motor timing tasks, on the other hand, require a timely motor response, with no sensory cue --- such as delayed blinking in response to a conditioned stimulus.
While some tasks can be considered exclusively motor, or sensory, most tasks possess both sensory and motor components, namely, reproducing a temporal pattern, e.g., a Morse code.
\section{Embodiment}
\label{ch:intro:Embodiment}
\epigraph{Je pense, donc je suis (I think, therefore I am).}
{\textit{ Ren\'{e} Descartes, Discours de la M\'{e}thode}}
\noindent
I am not invoking Descartes just because I am in France, there is a point too!
This quote implies a duality between the brain and the body: the reason I exist is my mind, not the body.
Although the delicacies of the \emph{mind-body problem} are out the scope of this work, a simple reading suggests that the brain is the ruler of the body.
This simple unidirectional approach has been vastly used in fields such as robotics, by designing agents with a central processing unit that commands the actuators.
This simplicity, however, comes at a cost.
The most unremarkable actions that animals perform with little cognitive load, such as grasping an object or locomotion on uneven terrains, have proved to be painstakingly difficult to implement in robots~\cite{Pfeifer2006Book}.
For decades now, an alternative approach has been proposed that has improved the performance of robotic agents~\cite{Brooks1991AI}.
\par
Since then, embodiment\footnotemark, has enabled engineering of more robust and adaptable robots, inspired from biological organisms.
\footnotetext{
    According to the Oxford dictionary, embodiment is defined as: ``A tangible or visible form of an idea, quality, or feeling".
    }
\citeauthor{Pfeifer2007Sci} present insect locomotion as a very convincing example of taking advantage of embodiment principles in robotics.
Insects demonstrate coordinated walking and running, which given their six legs, pose a challenging problem with dozens of degrees of freedom, in particular on uneven terrains.
It is plausible to assume they do not solve the inverse kinematic problem for all their joints at all the times, which was the classic approach in robotics and requires enormous computational resources.
However, by taking embodiment into account, pushing back a single leg, which could be detected by angle sensors in the joint, could command all the other joints to move in the ``correct" direction.
This way, a low level communication between the legs could be exploited to achieve leg coordination without any central controller in the nervous system~\cite{Pfeifer2007Sci}.
\par
In the animal kingdom, embodiment enables both cognition--even the most abstract processes, like mathematical reasoning~\cite{Lakoff2000Book}-- and action.
In this framework, behavior is not reduced to internal computations, rather it is the manifestation of intricate brain-body-environment interactions.
Perception of the external world relies upon how the information is channeled through different parts of the body and differences in the shape of body parts alters the incoming and outgoing signals~\cite{Gomez2019Neuron}.
The body also shapes the way we interact with our environment.
\Citeauthor{Gomez2019Neuron} discuss the interesting case of the well-coordinated stepping behavior in human infants~\cite{Gomez2019Neuron}.
When held upright, newborns show coordinated step-like movements.
This phenomenon disappears after around~2~months.
While it was long assumed that this is due to the developing nervous system, \citeauthor{Thelen1984InfBeh} showed that loss of stepping behavior is due to weight gain of the legs and it can be recovered by submerging the legs in water (which would decrease their mass)~\cite{Thelen1984InfBeh}.
Thus, embodiment, through brain-body-environment interactions subjects us to the laws of physics---having to deal with gravity, friction, and most relevant to this work, forward arrow of time~\cite{Pfeifer2006Book}.


\subsection{Embodied Clock}
\label{ch:intro:EmbodiedClock}
\epigraph{Time by itself does not exist\ldots It must not be claimed that anyone can sense time apart from the movement of things.}
{\textit{Lucretius, Book 1}}
\noindent
Principles of embodiment could be applied to the time-estimation problem as well.
All the sensorimotor processes that comprise embodiment (and indeed everything else!) unfold in time.
Especially, movement has long been associated with time estimation, as far as one study stating that ``timing is inexorably tied to movement"~\cite{Wiener2019eNeuro}.\footnotemark
\footnotetext{
    It is noteworthy that the devices we use to measure time mostly do so by moving objects in space. Also, we extensively use metaphors containing movement and space references when speaking of time (\textit{holidays are approaching}, \textit{time flies})~\cite{Winter2015Cortex}. 
    }
\par
As early as \citeyear{Skinner1948}, it has been reported that periodic reward delivery leads to `superstitious' behavior, i.e., performing stereotypical actions between consecutive deliveries of the reinforcer~\cite{Skinner1948}.
For example, one pigeon was conditioned to turn counter-clockwise in the cage two or three times between each reward delivery which was every 5~s, irrespective of the animal's behavior.
Each pigeon in this study developed such a unique behavior~\cite{Skinner1948}.
Similar phenomenon has been reported in many other species as well.
\Citeauthor{Wilson1953} trained rats to press a lever after progressively longer intervals (from 15~s to 30~s) to get a food pellet~\cite{Wilson1953}.
Rats slowly adjusted their lever presses to the scheduled interval, however, during the interval, they too engaged in a recognizable chain of behaviors that the authors called `collateral'.
These behaviors were also unique to each animal.
Interestingly, with increasing the interval between reward deliveries, more links were added to the chain of collateral behaviors~\cite{Wilson1953}.
Both studies mentioned above explain these behaviors as being accidentally reinforced by reward delivery, which would make them more probable to occur later, which in turn would strengthen their association with the reward~\cite{Killeen1988}.
Such a mechanism explains why these behaviors are unique to individual subjects.
Developing accidentally-reinforced behaviors could bring about repercussions.
\Citeauthor{Falk1971}, in a very enlightening article, discusses `adjunctive' behavior in food-deprived rats without any water deprivation~\cite{Falk1971}.
When exposed to intermittent food delivery during their daily test session (3~hr long), animals followed each food pellet intake with consumption of excessive amounts of water (up to half their body weight during the session) until the next food delivery, while almost no water was consumed during the rest of the day, despite being available ad libitum.
This form of adjunctive behavior persisted even after water consumption during the session was discouraged by punishment~\cite{Falk1971}.\footnotemark
\footnotetext{
    He then discusses that even though this behavior seems absurd (``heating a large quantity of room-temperature water to body heat and expelling it as copious urine is wasteful for an animal already pressed for energy stores by food deprivation"), in certain ecological settings, it might provide an adaptive response even with evolutionary advantages.
    }
\par
Modern technology has enabled synchronized video tracking of behaving animals.
In tasks in which reinforcement is contingent upon respecting time intervals, animals do not stay still, but they take advantage of the structure of their environment to develop stereotyped motor routines whose duration amounts to the temporal constraint of the task.
In one study, rats and pigeons, trained to discriminate a 12~s stimulus from a 6~s one, developed `collateral' behaviors.
Rats, during the stimulus, engaged in sniffing, rearing, grooming, and moving from one lever to another.
Similarly, birds displayed pecking, bobbing\footnotemark, wing flapping, and moving between the keys in their cage.
Quantifying these behaviors better predicted their temporal judgement than the passage of time~\cite{Fetterman1998BehProc}.
\footnotetext{
    For those unfamiliar with bird behavior (such as myself), \emph{bobbing} refers to the two-phase movement of the head in birds, most commonly seen during walking when they hold their head while moving the body forward and then thrust their head faster than their body.
    Watching YouTube clips is advised!
    }
In one of the rare studies with precise monitoring of behavior, \citeauthor{Gouvea2014} trained rats (and one mouse) to categorize an interval as shorter or longer than 1.5~s by pressing a lever, correspondingly.
Animals demonstrate highly stereotyped and idiosyncratic behavior during the interval.
Critically, their perceptual report was best predicted based on their behavior, even from early in the trial~\cite{Gouvea2014}.
Similar idiosyncratic embodied strategies were also used by rats trained to reproduce a 700~ms interval by waiting between successive lever presses.
\Citeauthor{Kawai2015} reported that animals developed very specific and reproducible limb movements to fill the required interval~\cite{Kawai2015}.
Earlier work from our lab also reported stereotypical use of embodied strategies, adapted to a dynamic environment, in a task in which rats learned to wait 7~s before approaching the reward delivery port~\cite{Rueda2015NN}.
\par
Humans, too, seem to resort to motor activity to estimate time.
Naturally, people tend to develop rhythmical movements of body parts (e.g., tapping fingers or feet, moving arms, and nodding heads) to perceive elapsed time~\cite{Merchant2016CurrOp}.
Around 97\% of adults default to counting as a time estimation strategy, and interestingly, in research, different sorts of measures has been employed to prevent use of counting in favor of a more \textit{pure} time estimation strategy~\cite{Rattat2012}.
Similarly, children as young as 7~years old, estimate suprasecond time intervals by counting~\cite{Wilkening1987, Rakitin1998}.
Although counting could be in their heads (i.e., not out loud), it is difficult to separate it from the repeated experiences of counting the passing seconds aloud in everyday life, which is a motor activity:
    a sequence of coordinated movements across respiratory, laryngeal and supraglottal articulatory systems.
Indeed, it has also been proposed that explicit perception of time may be constructed implicitly by associating the duration of an interval with its sensorimotor content~\cite{Coull2018}, e.g., 1~s is the time one takes to rock their head with a certain speed, or the time it takes to say 1001, 1002,~\ldots~in cardiac resuscitation.
Instructing human subjects to not use motor strategies or interfering with overt movements, lowers their performance in a variety of time estimation tasks~\cite{Morillon2017PNAS, Wiener2019eNeuro, Meegan2000, Rakitin1998, Fautrelle2015PlosOne, Monier2019DevSci}.
\par
% Overall, there is convincing evidence of beneficial impact of movement in time estimation.
% However, there is more to embodiment than movement.
% It is a common observation that perception of time is not isomorphic with the actual elapsed time measured by clocks.
% Contextual factors, such as size, distance, speed, as well as attentional, motivational, and emotional state warp temporal perception.

\subsection{Costs}
\label{ch:intro:cost}
Being subject to the rules of physics is a major implication of embodiment.
An animal with a physical body in the real world is motivated to obtain the rewards (for survival or gratification) as soon as possible (due to competition, uncertainty,\ldots) while minimizing the energy expenditure (since resources are limited).
\section{Implementation}
\label{ch:intro:implementation}
% why basal ganglia
% (maybe first the anatomy)
The renowned neuroscientist, David Marr (1945--1980), proposed three levels of analysis to understand a complex system.
First, the \emph{computational level}, describes the task and the goal that need be achieved.
Second, the \emph{algorithmic level}, specifies the procedures for manipulating the information associated with the computation.
Third, the \emph{implementation level}, characterizes how the to physically realize the algorithm~\cite{Willshaw2015Marr}.
\Citeauthor{Krakauer2017Neuron} in a perspective article that greatly influenced this work, present the following example~\cite{Krakauer2017Neuron}.
Understanding a flying bird could be achieved at three levels:
A bird attempts to \textit{fly} (level~1:~computation) by \textit{flapping} its wings (level~2:~algorithm) which is plausible due to aerodynamic properties of the \textit{feathers} (level~3:~implementation).
They then argue that the explanatory power of studying feathers alone is fundamentally restricted, evident by some birds that fly without feathers and some types of flight that does not require flapping.
As it pertains to the link between brain and behavior, it may be much more difficult to infer the algorithms used by brain from studying the nervous system, compared to understanding them at a computational level.
\par
Thus far, I portrayed the case for behavioral importance of time estimation (level~1), and different possible approaches to estimate an interval (dedicated, emergent, and embodied clock, level~2).
In this section, I will address how any of those could be implemented in the brain (level~3).
Of all the brain regions that have been suggested to be involved in time perception, across a wide range of tasks and scales, \gls{bg} is of unique interest.
For decades, \gls{bg} have been the focus of many timing studies~\cite[see][]{Paton2018NeuronRev}, as well as motor studies~\cite[see][]{Turner2010CurrOpinNeurobiol}.



\subsection{Basal Ganglia as a Clock}
\label{ch:intro:BGTime}
\epigraph{One may be inclined to state that researchers are actually clueless concerning the question of how the brain processes time.}
{\textit{Marc Wittmann, Nature Reviews Neuroscience, 2013}}
\noindent

Many brain structures have been proposed to contribute to time estimation.
The \gls{bg}, a set of interconnected subcortical nuclei, are especially of interest, since they are directly involved in motor processes as well~\cite{Grillner2015}.
Moreover, the \gls{bg} are also involved in reinforcement learning---selecting actions in an uncertain world in a way that maximizes reward in the long term~\cite{Petter2018}.
Such learning necessitates an understanding of temporal contingencies in order to maximize some future reward.
Behavioral data also supports that animals build probabilistic models for timing~\cite{li2013PNAS}.
In general, execution of any complex behavior requires proper timing of the comprising sub-actions.
\par
The \gls{bg} are often implicated in timescales of several hundreds of milliseconds to several seconds~\cite{Paton2018NeuronRev}.
Evidence of involvement of the \gls{bg} in timing stems from a variety of sources, including pathologies such as \gls{pd}, lesion studies, and pharmacological and genetic manipulations.
\par
Following the taxonomy discussed in \autoref{ch:intro:taxonomy}, there is some evidence of involvement of the \gls{bg} in sensory timing.
\Citeauthor*{Rao2001} reported encoding of time intervals in the human striatum in a task in which subjects reported whether an interval were shorter or longer than a standard interval of 1200~ms.\footnotemark\
They also observed a dynamic network of cortical activity in inferior parietal, premotor, and dorsolateral prefrontal cortex.
These nodes in the network were attributed to different components of temporal processing, respectively, attention, memory, and interval comparison.
They collectively concluded implication of the ``striatal dopaminergic neurotransmission in hypothetical internal timekeeping mechanisms"~\cite{Rao2001}.
\footnotetext{
    This paradigm is commonly referred to as ``interval categorization task".
    }
\Citeauthor*{Pouthas2005} also investigated interval categorization for two durations (450~ms and 1300~ms).
They observed ramping striatal activity during both intervals.
They concluded a direct role of the basal ganglia in duration estimation, and that the caudate nucleus ``may support a clock mechanism"~\cite{Pouthas2005}.
Similar evidence exist in other species too.
\Citeauthor*{Gouvea2015Elife} trained rats in a sensory categorization task to judge whether an interval is shorter or longer than 1.5~s.
They decoded animals' choice and elapsed time from ensembles of striatal neuronal activity, whereas apparent behavior in an overhead video failed to do so.
Transient inactivation of the \gls{ds} impairs performance, however, it doesn't cause a systematic under-- or over--estimation~\cite{Gouvea2015Elife}.
\par
% Furthermore, the \gls{bg} are also well studied for their role in motor timing.
% In a range of disorders affecting the \gls{bg}, including \gls{pd}, Huntington's disease, Tourette's syndrome, drug abuse, and attention deficit disorder, an altered perception of time, or temporal patterns have been reported~\cite[see][]{Paton2018NeuronRev}.
Furthermore, the \gls{bg} are also well studied for their role in motor timing.
\Citeauthor*{Matell2003} trained rats to receive a reward in a fixed interval reinforcement schedule.\footnotemark\
The interval alternated between 10~s (25\% of trials) and 40~s (75\% of trials).
After learning, animals increased their lever press rate around the reinforced intervals.
Electrophysiological recordings from the striatum show neurons with tuned firing rate only around 10~s interval, but not 40~s, while apparent behavior of the animals is similar.
The authors then suggest that a population of duration-coding cells, each tune to different values, could accurately represent the elapsed time~\cite{Matell2003}.
\footnotetext{
    In operant conditioning, fixed interval reinforcement schedule refers to a type of conditioning whereby a response is reinforced (i.e., rewarded) only if a certain period of time (i.e., interval) has elapsed.
    }
\Citeauthor*{Mello2015} also used a similar task for intervals ranging between 12~s to 60~s.
They found striatal cells that rescaled their activity when intervals changed.
As rats adjusted to the new interval, time estimations decoded form population dynamics predicted animals' timing performance.
In another study, \Citeauthor*{Bakhurin2017JNeuro} used a conditioning paradigm to signal delayed reward delivery (2.5~s after cue onset).
Individual neurons recorded in the striatum and orbitofrontal cortex display sequential activity during the interval.
A machine learning algorithm was then trained to decode the elapsed time from the stimulus onset.
They show that both striatal and cortical networks ``encoded time, but the striatal network outperformed the orbitofrontal cortex".
Interestingly, removing the neurons modulated by licking activity from the decoder significantly reduced its performance, however, it still remained higher than chance level~\cite{Bakhurin2017JNeuro}.
\par
Another source of impact in the \glsentrylong{bg} is the neuromodulatory effect of \gls{da}.
\Glsentrylong{da}'s role in reward processing and circuit dynamics of the striatum will be discussed later in sections~\ref{intro:BGMotor} and \ref{intro:BGAnatomy}.
However, \gls{da} is also believed to be involved in timing~\cite{Paton2018NeuronRev}.
In a peak interval procedure\footnotemark, \Citeauthor*{DeCorte2019} found that \gls{d2} blockade delayed start and stop times for an interval of 6~s.
Whereas, blockade of \glspl{d1} delayed stop times.
Then they stress the role of the \gls{ds} in timing, with \gls{da} ``being particularly critical for the temporal control of action"~\cite{DeCorte2019}.
\footnotetext{
    Peak interval procedure is a common task used to study timing.
    Similar to fixed interval schedules, a cue indicates that a response will be reinforced only after a certain period of time has elapsed.
    The profile of the response around the interval is then studied.
    }
\Glsentrylong{da} neurons encode reward prediction errors which requires accurate reward predictions~\cite[see][]{Berke2018NN}.
\Citeauthor*{Takahashi2016} recorded from \gls{da} neurons of rats while they performed a task with uncertainty in reward timing and reward number.
Neuronal activity showed error signals in response to both types of prediction error, however, after ventral striatal lesions, neurons only responded to changes in reward number, and not reward timing.
These results suggest that time-dependant component of reward prediction of \gls{da} neurons might rely on the ventral striatum~\cite{Takahashi2016}.
In an interesting study, \Citeauthor*{Paton2016Sci} measured and manipulated the activity of \gls{da} neurons in a 1.5~s interval categorization task.
\Gls{da}ergic activity predicted animal's time estimates.
Transient activation/inhibition of \gls{da} neurons caused under-/over-estimation of the interval.
Hence, they concluded that ``\gls{da} neurons, which are so central to reward processing, exert control over time estimation"~\cite{Paton2016Sci}, although these results reflect \gls{da} function in general, not specifically in the \gls{bg}.
Similar to scaling of neuronal activity in the striatum~\cite{Mello2015}, \gls{da} concentration in the \gls{ds} is also scalable to time intervals in several second time range~\cite{Howard2017}.
However, \citeauthor{Howard2017} then conducted a series of experiments and concluded that the \gls{da} signal in the \gls{ds} does not reflect interval timing \textit{per se}, rather it is specific to behavioral choice of action~\cite{Howard2017}.




\subsection{Basal Ganglia as a Sensorimotor System}
\label{intro:BGMotor}
\epigraph{Why do we and other animals have brains?\ldots You may reason that we have one to perceive the world or to think, and that is completely wrong\ldots We have a brain for one reason and one reason only, and that is to produce adaptable and complex movements.}
{\textit{Daniel Wolpert, TED talk}}
\noindent

Here, I am supposed to review the literature on the function of the \gls{bg}.

\subsubsection{Motor Control} \label{intro:motorControl}
\subsubsection{Motor Learning} \label{intro:motorLearning}
% \subsubsection{DLS}
% \subsubsection{DMS}
\subsubsection{Action Selection} \label{intro:actionSelection}
\subsubsection{Motor Habits} \label{intro:motorHabits}
\subsubsection{Cognition} \label{intro:bgCognition}

\subsection{Pathologies}
\label{intro:Motor:pathologies}

\section[Motivation and Organization]{Motivation, and the Organization of the Thesis}
\label{intro:question}

The work presented in this thesis has two fronts that although might seem unrelated at the first glance, are conceptually connected.
First is concerned with the question of how animals often act as though they have a sense of time.
Enormous body of experimental and theoretical research implicates plenty of brain areas as providers of a time signal.
Such a mechanism could be affected by external factors (e.g., reward rate and motivation), however, it is usually assumed to be the means by which well-timed actions are generated.
This is what I call ``internal time estimation", not that the world exterior to the brain is irrelevant, but meaning that the brain has a sense of time that underlies behavior.
Alternatively, we hypothesized that there is no sense of time per se, and that time is perceived through interactions with the environment.
In other words, the duration of an interval is displaced by its sensorimotor content.
Such ``embodied time estimation" provides a much more parsimonious explanation, and is in alignment with the long-reported and replicated observation across many species that animals produce stereotyped motor sequences under temporal constraints.
Nonetheless, this hypothesis has not been very popular!
Perhaps partly due to technological limitations to monitor a wide range of animal behavior (in rodents, from locomotion to whisking and sniffing), especially in standard experimental paradigms inside Skinner boxes; and in my opinion, partly due to a general brain-centric view where the brain is the puppeteer of the body.
