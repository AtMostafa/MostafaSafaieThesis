\section[Time Taxonomy]{Time Taxonomy} \label{ch:intro:taxonomy}
It is important to point out different categories of tasks used to study timing.\footnotemark
Appropriate classification of a phenomenon, alone, could lead to scientific advances.
\footnotetext{
    This section follows the arguments presented by~\citeauthor{Paton2018NeuronRev} in~\cite{Paton2018NeuronRev}.
    }
First step toward a taxonomy of time is to define what could be considered a timing task.
Not every task with a temporal dependency is regarded as a time estimation task.
A timing task requires an explicit understanding of a given duration, i.e., one would need a clock to solve the task.
For example, judging which of any two sensory stimuli occurred first does not require a timing device to solve and hence, is not a timing task.
On the other hand, judging which of those stimuli were longer, indeed is a timing task, since it cannot be solved without any reference for time.
\par
It is not perfectly clear, but there is some consensus over principal dimensions of the taxonomy of time.

\paragraph{Subsecond vs.\ Suprasecond Timing.} \label{ch:intro:taxonomy:SUBvsSUPRA}
There is ample evidence that timing relies on different mechanisms for short and long timescales~\cite[see][]{Paton2018NeuronRev}.
Although the boundary is not definite, for timescales relevant to this work, short intervals are several tens of milliseconds ($50-100$~ms), and long intervals include several hundred milliseconds to several seconds.
% Interestingly, subsecond intervals are closer to natural rhythmicities of the body, e.g., respiration and heart beat.

\paragraph{Interval vs.\ Pattern Timing.} \label{ch:intro:taxonomy:INTvsPAT}
There is evidence of differential neural mechanisms at play for simple timing tasks (such as reproducing a duration) as  opposed to tasks where the global temporal structure of the stimuli is determinant (such as recognizing the tempo of a song)~\cite{teki2011}.

\paragraph{Sensory vs.\ Motor Timing.} \label{ch:intro:taxonomy:SENvsMOT}
This dimension of time taxonomy, not unlike the other two, is a continuum.
In sensory tasks the subject analyzes the temporal information in the external world and reports their decision, such as an interval discrimination task.
Motor timing tasks, on the other hand, require a timely motor response, with no sensory cue --- such as delayed blinking in response to a conditioned stimulus.
While some tasks can be considered exclusively motor, or sensory, most tasks possess both sensory and motor components, namely, reproducing a temporal pattern, e.g., a Morse code.