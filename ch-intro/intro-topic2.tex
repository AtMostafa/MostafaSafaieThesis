\section{Embodiment}
\label{ch:intro:Embodiment}
\epigraph{Je pense, donc je suis (I think, therefore I am).}
{\textit{ Ren\'{e} Descartes, Discours de la M\'{e}thode}}
\noindent
I am not invoking Descartes just because I am in France, there is a point too!
This quote implies a duality between the brain and the body: the reason I exist is my mind, not the body.
Although the delicacies of the \emph{mind-body problem} is out the scope of this work, a simple reading suggests that the brain is the ruler of the body.
This simple unidirectional approach has been vastly used in fields such as robotics, by designing agents with a central processing unit that commands the actuators.
This simplicity, however, comes at a cost.
The most unremarkable actions that animals perform with little cognitive load, such as grasping an object or locomotion on uneven terrains, have proved to be painstakingly difficult to implement in robots~\cite{Pfeifer2006Book}.
For decades now, an alternative approach has been proposed that has improved the performance of robotic agents~\cite{Brooks1991AI}.
\par
Since then, embodiment\footnotemark, has enabled engineering of more robust and adaptable robots, inspired from biological organisms.
\footnotetext{
    According to the Oxford dictionary, embodiment is defined as: ``A tangible or visible form of an idea, quality, or feeling".
    }
\citeauthor{Pfeifer2007Sci} present insect locomotion as a very convincing example of taking advantage of embodiment principles in robotics.
Insects demonstrate coordinated walking and running, which given their six legs, pose a challenging problem with dozens of degrees of freedom, in particular on uneven terrains.
It is plausible to assume they do not solve the inverse kinematic problem for all their joints at all the times, which was the classic approach in robotics and requires enormous computational resources.
However, by taking embodiment into account, pushing back a single leg, which could be detected by angle sensors in the joint, could command all the other joints to move in the ``correct" direction.
This way, a low level communication between the legs could be exploited to achieve leg coordination without any central controller in the nervous system~\cite{Pfeifer2007Sci}.
\par
In the animal kingdom, embodiment enables both cognition--even the most abstract processes, like mathematical reasoning~\cite{Lakoff2000Book}-- and action.
In this framework, behavior is not reduced to internal computations, rather it is the manifestation of intricate brain-body-environment interactions.
Perception of the external world relies upon how the information is channeled through different parts of the body and differences in the shape of body parts alters the incoming and outgoing signals~\cite{Gomez2019Neuron}.
The body also shapes the way we interact with our environment.
\Citeauthor{Gomez2019Neuron} discuss the interesting case of the well-coordinated stepping behavior in human infants~\cite{Gomez2019Neuron}.
When held upright, newborns show coordinated step-like movements.
This phenomenon disappears after around~2~months.
While it was long assumed that this is due to the developing nervous system, \citeauthor{Thelen1984InfBeh} showed that loss of stepping behavior is due to weight gain of the legs and it can be recovered by submerging the legs in water (which would decrease their mass)~\cite{Thelen1984InfBeh}.
Thus, embodiment, through brain-body-environment interactions subjects us to the laws of physics---having to deal with gravity, friction, and most relevant to this work, forward arrow of time~\cite{Pfeifer2006Book}.


\subsection{Embodied Clock}
\label{ch:intro:Embodiment:Embodied Clock}
\epigraph{Time by itself does not exist\ldots It must not be claimed that anyone can sense time apart from the movement of things.}
{\textit{Lucretius, Book 1}}
\noindent
Principles of embodiment could be applied to time-estimation problems as well.
All the sensorimotor processes that comprise embodiment (and indeed everything else!) unfold in time.
Especially, movement has long been associated with time estimation, as far as one study stating that ``timing is inexorably tied to movement"~\cite{Wiener2019eNeuro}.
\par
As early as \citeyear{Skinner1948}, it has been reported that periodic reward delivery leads to `superstitious' behavior, i.e., performing stereotypical actions between consecutive deliveries of the reinforcer~\cite{Skinner1948}.
For example, one pigeon was conditioned to turn counter-clockwise in the cage two or three times between each reward delivery which was every 5~s, irrespective of the animal's behavior.
Each pigeon in this study developed such a unique behavior~\cite{Skinner1948}.
Similar phenomenon has been reported in many other species as well.
\Citeauthor{Wilson1953} trained rats to press a lever after progressively longer intervals (from 15~s to 30~s) to get a food pellet~\cite{Wilson1953}.
Rats slowly adjusted their lever presses to the scheduled interval, however, during the interval, they too engaged in a recognizable chain of behaviors that the authors called `collateral'.
These behaviors were also unique to each animal.
Interestingly, with increasing the interval between reward deliveries, more links were added to the chain of collateral behaviors~\cite{Wilson1953}.
Both studies mentioned above explain these behaviors as being accidentally reinforced by reward delivery, which would make them more probable to occur later, which in turn would strengthen their association with the reward~\cite{Killeen1988}.
Such a mechanism explains why these behaviors are unique to individual subjects.
Developing accidentally-reinforced behaviors could bring about repercussions.
\Citeauthor{Falk1971}, in a very enlightening article, discusses `adjunctive' behavior in food-deprived rats without any water deprivation~\cite{Falk1971}.
When exposed to intermittent food delivery during their daily test session (3~hr long), animals followed each food pellet intake with consumption of excessive amounts of water (up to half their body weight during the session) until the next food delivery, while almost no water was consumed during the rest of the day, despite being available ad libitum.
This form of adjunctive behavior persisted even after water consumption during the session was discouraged by punishment~\cite{Falk1971}.\footnotemark
\footnotetext{
    He then discusses that even though this behavior seems absurd (``heating a large quantity of room-temperature water to body heat and expelling it as copious urine is wasteful for an animal already pressed for energy stores by food deprivation"), in certain ecological settings, it might provide an adaptive response even with evolutionary advantages.
    }
\par
Modern technology has enabled synchronized video tracking of behaving animals.
In tasks in which reinforcement is contingent upon respecting time intervals, animals do not stay still, but they take advantage of the structure of their environment to develop stereotyped motor routines whose duration amounts to the temporal constraint of the task.
In one study, rats and pigeons, trained to discriminate 12~s stimulus from 6~s, developed `collateral' behaviors.
Rats, during the stimulus, engaged in sniffing, rearing, grooming, and moving from one lever to another.
Similarly, birds displayed pecking, bobbing\footnotemark, wing flapping, and moving between the keys in their cage.
Quantifying these behaviors better predicted their temporal judgement than the passage of time~\cite{Fetterman1998BehProc}.
\footnotetext{
    For those unfamiliar with bird behavior (such as myself), \emph{bobbing} refers to the two-phase movement of the head in birds, most commonly seen during walking when they hold their head while moving the body forward and then thrust their head faster than their body.
    Watching YouTube clips is advised!
    }
In one of the rare studies with precise monitoring of behavior, \citeauthor{Gouvea2014} trained rats (and one mouse) to categorize an interval as shorter or longer than 1.5~s by pressing a lever, correspondingly.
Animals demonstrate highly stereotyped and idiosyncratic behavior during the interval.
Critically, their perceptual report was best predicted based on their behavior, even from early in the trial~\cite{Gouvea2014}.
Similar idiosyncratic embodied strategies were also used by rats trained to reproduce a 700~ms interval by waiting between successive lever presses.
\Citeauthor{Kawai2015} reported that animals developed very specific and reproducible limb movements to fill the required interval~\cite{Kawai2015}.
Earlier work from our lab also reported stereotypical use of embodied strategies, adapted to a dynamic environment, in a task in which rats learned to wait 7~s before approaching the reward delivery port~\cite{Rueda2015NN}.
\par
Humans, too, seem to resort to motor activity to estimate time.
Naturally, people tend to develop rhythmical movements of body parts (e.g., tapping fingers or feet, moving arms, and nodding heads) to perceive elapsed time~\cite{Merchant2016CurrOp}.
Similarly, humans, as young as 7~years old, estimate suprasecond time intervals by counting~\cite{Wilkening1987, Rakitin1998}.
Although counting could be in their heads (i.e., not out loud), it is difficult to separate it from the repeated experiences of counting the passing seconds aloud in everyday life, which is a motor activity:
    a sequence of coordinated movements across respiratory, laryngeal and supraglottal articulatory systems.
Around 97\% of adults default to counting as a time estimation strategy, and interestingly, in research, different sorts of measures has been employed to prevent use of counting in favor of a more \textit{pure} time estimation strategy~\cite{Rattat2012}.\footnotemark
\footnotetext{
    It is noteworthy that the devices we use to measure time mostly do so by moving objects in space. Also, we extensively use metaphors containing movement and space references when speaking of time (\textit{holidays are approaching}, \textit{time flies})~\cite{Winter2015Cortex}. 
    }
There is convincing evidence of beneficial impact of movement in time estimation.
Instructing human subjects to not use motor strategies, or otherwise interfering with overt movements, lowers their performance in a variety of time estimation tasks~\cite{Morillon2017PNAS, Wiener2019eNeuro, Meegan2000, Rakitin1998, Fautrelle2015PlosOne, Monier2019DevSci}.










\subsection{Cost}
\label{ch:intro:Embodiment:cost}



% There is convincing experimental evidence that our perception of time is not isomorphic with the time measured by human-made clocks and strongly depends on contextual factors affecting the subjects' attentional, motivational or emotional state~\cite{Paton2016Sci, Coull2004Sci, Droit2007TICS, Effron2006Emotion, Pariyadath2007Plos, Gable2012PsychSci}.\footnotemark
% \footnotetext{
%     Time flies when we are fully engaged in an activity (giving a 20 minutes-long neuroscience talk) but can terribly drag when we are bored (listening to a 20 minutes-long neuroscience talk).
%     }
% Both emergent and dedicated algorithms for time estimation account for the state-dependant variance in time perception by a neuromodulator-based slowing or speeding of neuronal clocks~\cite{Paton2016Sci, Simen2016Sci}.
% Alternatively, the strong influence of contextual factors on time perception may indicate that mechanisms beyond internal neural activity play a primary role in how humans and animals perceive time and adapt their behavior accordingly.
% Indeed, the embodiment framework proposes that important aspects of cognition are determined by physical features of the body, which in turn influence how humans or animals interact with their environment~\cite{Pfeifer2006Book}.
% That is to say, animals (and humans) may primarily replace the abstract concept of having to estimate a certain duration with an embodied strategy that happens to last the same duration.
% In tasks whereby a fixed time interval must be respected to obtain a reward, animals do not stay immobile, rather they take advantage of the structure of their environment to develop ritualistic motor routines whose durations match these task-specific temporal constraints~\cite{Gouvea2014, Kawai2015, Rueda2015NN}.
% Although humans can seemingly estimate time, either by counting in their heads or otherwise, it is difficult to separate this mental faculty from memories of counting the passing seconds aloud, which in turn is a sequence of coordinated movements across respiratory, laryngeal and supraglottal articulatory systems, or in general, any motor memory~\cite{Coull2018}.
% In addition, humans naturally tend to estimate time intervals through rhythmical movements of some of their body parts (by flicking fingers, moving arms, nodding heads or tapping feet).\footnotemark
% \footnotetext{
%     It is noteworthy that, to reliably estimate time, humans have created clocks, that indicate time by moving objects in space and extensively use metaphors containing movement and space references when speaking of time (\emph{holidays are approaching}, \emph{time flies})~\cite{Nunez2013TICS,Winter2015Cortex}.
%     }
% It has also been recently argued that explicit perception of time may be constructed implicitly through association between the duration of an interval and its sensorimotor content~\cite{Coull2018}.
% Thus, animals (and humans) may primarily rely on the movement of their body in space to estimate time~\cite{VoletRev2013}, which incidentally explains why brain regions involved in movement control, such as the \gls{bg}, have consistently been associated with time perception~\cite{Coull2018}. 







