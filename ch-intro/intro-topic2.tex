\section{Embodiment}
\label{ch:intro:Embodiment}
\epigraph{Je pense, donc je suis (I think, therefore I am).}
{\textit{ Ren\'{e} Descartes, Discours de la M\'{e}thode}}
\noindent
I am not invoking Descartes just because I am in France, there is a point too!
This quote implies a duality between the brain and the body: the reason one exists is one's mind, not the body.
Although the intricacies of the \textit{mind-body problem} are not the focus of this work, a simple reading suggests that the brain is the ruler of the body.
This simple unidirectional approach has been vastly used in fields such as robotics, by designing agents with a central processing unit that commands the actuators.
This simplicity, however, comes at a cost.
The most unremarkable actions that animals perform with little cognitive load, such as grasping an object or locomotion on uneven terrains, have proved to be painstakingly difficult to implement in robots~\cite{Pfeifer2006Book}.
For decades now, an alternative approach has been proposed that has improved the performance of robotic agents~\cite{Brooks1991AI}.
\par
Since then, embodiment,\footnotemark\ has enabled engineering of more robust and adaptable robots, inspired from biological organisms.
\footnotetext{
    According to the Oxford dictionary, embodiment is defined as: ``A tangible or visible form of an idea, quality, or feeling''.
    }
\citeauthor{Pfeifer2007Sci} present insect locomotion as a very convincing example of taking advantage of embodiment principles in robotics~\cite{Pfeifer2007Sci}.
Insects demonstrate coordinated walking and running, which given their six legs, pose a challenging problem with dozens of degrees of freedom, in particular on uneven terrains.
It is plausible to assume they do not solve the kinematic problem for all their joints at every moment, which was the classic approach in robotics and required enormous computational resources.
However, by taking embodiment into account, pushing back a single leg, which could be detected by angle sensors in the joint, could command all the other joints to move in the `correct' direction.
This way, a low level communication between the legs could be exploited to achieve leg coordination without any central controller in the nervous system~\cite{Pfeifer2007Sci}.
\par
In the animal kingdom, embodiment enables both cognition--even the most abstract processes, like mathematical reasoning~\cite{Lakoff2000Book}-- and action.
In this framework, behavior is not reduced to internal computations, rather it is the manifestation of intricate brain-body-environment interactions.
Perception of the external world relies upon how the information is channeled through different parts of the body and differences in the shape of body parts alter the incoming and outgoing signals~\cite{Gomez2019Neuron}.
The body also shapes the way we interact with our environment.
\Citeauthor{Gomez2019Neuron} discuss the interesting case of the well-coordinated stepping behavior in human infants~\cite{Gomez2019Neuron}.
When held upright, newborns show coordinated step-like movements.
This phenomenon disappears after around~2~months.
While it was long assumed that this is due to the developing nervous system, \citeauthor{Thelen1984InfBeh} showed that loss of stepping behavior is due to weight gain of the legs and it can be recovered by submerging the legs in water (which would decrease their mass)~\cite{Thelen1984InfBeh}.
Thus, embodiment, through brain-body-environment interactions subjects us to the laws of physics---having to deal with gravity, friction, and inertia~\cite{Pfeifer2006Book}.


\subsection{Embodied Clock}
\label{ch:intro:EmbodiedClock}
\epigraph{Time by itself does not exist\ldots It must not be claimed that anyone can sense time apart from the movement of things.}
{\textit{Lucretius, Book 1}}
\noindent
Principles of embodiment could be applied to the time-estimation problem as well.
All the sensorimotor processes that comprise embodiment (and indeed everything else!) unfold in time.
Especially, movement has long been associated with time estimation, so far as one study stating that ``timing is inexorably tied to movement''~\cite{Wiener2019eNeuro}.\footnotemark
\footnotetext{
    It is noteworthy that the devices we use to measure time mostly do so by moving objects in space.
    Also, we extensively use metaphors containing movement and space references when speaking of time (\textit{holidays are approaching}, \textit{time flies})~\cite{Winter2015Cortex}.
    }
\par
As early as \citeyear{Skinner1948}, it has been reported that periodic reward delivery leads to `superstitious' behavior, i.e., performing stereotypical actions between consecutive deliveries of the reinforcer~\cite{Skinner1948}.
For example, one pigeon was conditioned to turn counter-clockwise in the cage two or three times between each reward delivery which was every 5~s, irrespective of the animal's behavior.
Each pigeon in this study developed such a unique behavior~\cite{Skinner1948}.
Similar phenomenon has been reported in many other species as well.
\Citeauthor{Wilson1953} trained rats to press a lever after progressively longer intervals (from 15~s to 30~s) to get a food pellet~\cite{Wilson1953}.
Rats slowly adjusted their lever presses to the scheduled interval however, during the interval, they also engaged in a recognizable chain of behaviors that the authors called `collateral'.
These behaviors were unique to each animal too.
Interestingly, with increasing the interval between reward deliveries, more links were added to the chain of collateral behaviors~\cite{Wilson1953}.
Both studies mentioned above explain these behaviors as being accidentally reinforced by the reward delivery, which would make them more probable to occur later, which in turn would strengthen their association with the reward~\cite{Killeen1988}.
Such a mechanism explains why these behaviors are unique to individual subjects.
Developing accidentally-reinforced behaviors could bring about repercussions.
\Citeauthor{Falk1971}, in a very enlightening article, discusses `adjunctive' behavior in food-deprived rats without any water deprivation~\cite{Falk1971}.
When exposed to intermittent food delivery during their daily test session (3~hr long), animals followed each food pellet intake with consumption of excessive amounts of water (up to half their body weight) until the next food delivery, while almost no water was consumed during the rest of the day, despite being available ad libitum.
This form of adjunctive behavior persisted even after water consumption during the session was discouraged by punishment~\cite{Falk1971}.\footnotemark
\footnotetext{
    He then discusses that even though this behavior seems absurd (``heating a large quantity of room-temperature water to body heat and expelling it as copious urine is wasteful for an animal already pressed for energy stores by food deprivation''), in certain ecological settings, it might provide an adaptive response even with evolutionary advantages.
    }
\par
Modern technology has enabled synchronized video tracking of behaving animals.
In tasks in which reinforcement is contingent upon respecting time intervals, animals do not stay still, but they take advantage of the structure of their environment to develop stereotyped motor routines whose duration amounts to the temporal constraint of the task.
In one study, rats and pigeons, trained to discriminate a 12~s stimulus from a 6~s one, developed `collateral' behaviors.
Rats, during the stimulus, engaged in sniffing, rearing, grooming, and moving from one lever to another.
Similarly, birds displayed pecking, bobbing,\footnotemark\ wing flapping, and moving between the keys in their cage.
Quantifying these behaviors better predicted their temporal judgment than the passage of time~\cite{Fetterman1998BehProc}.
\footnotetext{
    For those unfamiliar with bird behavior (such as myself), \textit{bobbing} refers to the two-phase movement of the head in birds, most commonly seen during walking when they hold their head while moving the body forward and then thrust their head faster than their body.
    Watching YouTube clips is advised!
    }
In another study with precise monitoring of behavior, \citeauthor{Gouvea2014} trained rats (and one mouse) to categorize an interval as shorter or longer than 1.5~s by pressing a lever, correspondingly.
Animals demonstrated highly stereotyped and idiosyncratic behavior during the interval.
Critically, their perceptual report was best predicted based on their behavior, even from early in the trial~\cite{Gouvea2014}.
Similar idiosyncratic embodied strategies were also used by rats trained to reproduce a 700~ms interval by waiting between successive lever presses.
\Citeauthor{Kawai2015} reported that animals developed very specific and reproducible limb movements to fill the required interval~\cite{Kawai2015}.
Earlier work from our group also reported stereotypical use of embodied strategies, adapted to a dynamic environment, in a task in which rats learned to wait 7~s before approaching the reward delivery port~\cite{Rueda2015NN}.
\par
Humans, too, seem to resort to motor activity to estimate time.
Naturally, people tend to develop rhythmical movements of body parts (e.g., tapping fingers or feet, moving arms, and nodding heads) to perceive elapsed time~\cite{Merchant2016CurrOp}.
Around 97\% of adults default to counting as a time estimation strategy, and interestingly, in research, different sorts of measures has been employed to prevent use of counting in favor of a more \textit{pure} time estimation strategy~\cite[see][]{Rattat2012}.
Similarly, children as young as 7~years old, estimate suprasecond time intervals by counting~\cite{Wilkening1987, Rakitin1998}.
Although counting could be in their heads (i.e., not out loud), it is difficult to separate it from the repeated experiences of counting the passing seconds aloud in everyday life, which is a motor activity:
    a sequence of coordinated movements across respiratory, laryngeal and supraglottal articulatory systems.
Indeed, it has also been proposed that explicit perception of time may be constructed implicitly by associating the duration of an interval with its sensorimotor content~\cite{Coull2018}.
For instance, 1~s is the time one takes to rock their head with a certain speed, or the time it takes to say 1001, 1002,~\ldots~in cardiac resuscitation.
Instructing human subjects to not use motor strategies or interfering with overt movements, lowers performance in a variety of time estimation tasks~\cite{Morillon2017PNAS, Wiener2019eNeuro, Meegan2000, Rakitin1998, Fautrelle2015PlosOne, Monier2019DevSci}.


\subsection{Costs of Embodiment}
\label{ch:intro:cost}
Being subject to the laws of physics is a major implication of embodiment.
An animal with a physical body in the real world needs to obtain the rewards (for survival or gratification) as soon as possible (due to competition, uncertainty,\ldots) while minimizing the energy expenditure (since resources are limited).
Foraging is a relevant example.
A honeybee harvests the nectar of a flower for a certain \textit{duration}.
At some point, perhaps following a diminishing rate of supply, it decides to leave the flower in order to find another one and flies off with a certain \textit{speed}.
These behaviors are well-predicted by theories such as \emph{optimal foraging} in a diverse group of species, from worms to humans~\cite{Yoon2018PNAS}.
Optimal foraging proposes a kind of `currency' with evolutionary advantages to behave in a way that it become maximized~\cite{Shadmehr2019TINS,Carland2019NeuroSci}.
This currency is the \emph{capture rate} and in principle, it is defined as the sum of the acquired rewards,\footnotemark\ minus costs of action, divided by total elapsed time~\cite{Shadmehr2019TINS}.
\footnotetext{
    Reward itself could be considered as a function of economic utility, and the certainty with which the action yields the reward.
    \Citeauthor{Shadmehr2019TINS} defined \textit{economic utility} as ``a measure of how much one values a particular good'', i.e., the subjective value of outcome~\cite{Shadmehr2019TINS}.
    % Reward evaluation and its impact on choice is a topic in decision-making and out of the scope of this work.
    }
\par
Maximizing the capture rate is also an arguably intuitive policy in the case of the time estimation problem, since animals naturally use motor strategies to fill the interval that they need to estimate.
Thus, time estimation transforms to performing a motor routine with the following properties:
\begin{itemize}[noitemsep]
    \item It is of appropriate duration and reproducible to generate reliable well-timed responses;
    \item It is the least costly.
\end{itemize}
Although the capture rate ostensibly depends on three parameters, in practice those parameters are not mutually independent.
For instance, the cost of action mostly translates to the metabolic cost, which is directly related to the speed of movement.
The faster the speed, the higher the metabolic cost, and therefore the lower the capture rate.
However, faster movements finish earlier, i.e., shorter elapsed time, and therefore higher capture rates!
So, one way by which defining parameters of the capture rate become interdependent is the passage of time itself.


\subsubsection{Cost of Time}
The passage of time inevitably incurs a cost to the subjective value of the reward.
For example, young adults prefer a small amount of money immediately, rather than a larger sum in a year.
This common attitude is referred to as \emph{temporal discounting}\!{}.
Children are known to discount rewards more quickly and the elderly, more slowly.
The rate with which one discounts future rewards varies among individuals and is used as a measure of impulsivity, i.e., higher rate of discount means more impulsive behavior~\cite{Choi2014JNeurosci}.
The discounting of the reward value is usually characterized via a hyperbolic function of time.
\par
Strikingly, in humans and other animals and across a wide range of tasks, there is a correlation between discounting of reward and control of movements~\cite{Shadmehr2010Jneurosci, Choi2014JNeurosci, Berret2018SciReports, Shadmehr2016CurrBiol, Berret2016JNeurosci}.
Individuals with naturally faster movements discount future rewards more steeply~\cite{Choi2014JNeurosci}.
Moreover, animals move faster when the prospect of a greater amount of reward exists.
For instance, in an environment with a higher reward rate, monkeys in a decision-making task, chose the target with shorter deliberation times and faster saccade velocities~\cite{Thura2014JNeurosci}.
This phenomenon is remarkable since it bridges between the fields of decision-making and motor control.
It has been hypothesized that, in principle, the purpose of any goal-directed movement is transitioning to a more rewarding state, then, due to temporal discounting of rewards, the duration of movement per~se incurs a cost by postponing the reward acquisition~\cite{Shadmehr2010Jneurosci}.
This is called the \gls{cot} hypothesis.
\par
The concept of \gls{cot} has been applied for understanding why we don't move slower~\cite{Berret2016JNeurosci}.
Indeed, humans are extremely reluctant to move their arms slowly~\cite{Berret2018SciReports}, even though moving fast is constrained by energetic demand~\cite{Long2013RoSocInterface} and speed/accuracy trade-off~\cite{Harris2006BioCyber}.
Optimal control framework has been utilized to infer the shape of the \gls{cot} as a function of movement time.
Consistent with the empirical data, \gls{cot} displays a sigmoidal shape over relevant time durations~\cite{Berret2016JNeurosci}.
Moreover, \gls{cot} also accounts for inter-individual differences in \emph{vigor}\!.{}\footnotemark\
\footnotetext{
    Vigor is a key parameter of any movement.
    It is often correlated with several measurements of movement kinematics, such as speed and amplitude~\cite{Yttri2018MovDisorder}.
    Movement vigor is generally considered as those aspects of movement kinematics which are subject to motivational state, e.g., implicit motivation~\cite{Dudman2016CurrOpinNeurobiol}.
    }
\Citeauthor{Berret2018SciReports} show that in a single-joint self-paced arm reaching task, up to 89\% of inter-individual variability of vigor is explained by parameters of the \gls{cot} function, e.g., the value of reward~\cite{Berret2018SciReports}.
Similarly, delaying the reward, and thus decreasing its value, is associated with decreased saccade vigor~\cite{Shadmehr2010Jneurosci}.
Moreover, when human subjects are presented with two options with different values, the relative saccade vigor to each option reflects the subjective evaluation of the value of that option~\cite{Reppert2015JNeurosci}.
These results suggest that the expected reward upon action completion is an important determinant of the vigor with which the action is executed~\cite[see][for a review]{Shadmehr2019TINS}.
