\section{Embodiment}
\label{ch:intro:Embodiment}
\epigraph{Je pense, donc je suis (I think, therefore I am).}
{\textit{ Ren\'{e} Descartes, Discours de la M\'{e}thode}}
\noindent
I am invoking Descartes not just because I am in France, there is a point too!
This quote implies a duality between the brain and the body: the reason I exist is my mind, not the body.
Although the delicacies of the \emph{mind-body problem} is out the scope of this work, a simple reading suggests that the brain is the ruler of the body.
This simple unidirectional approach has been vastly used in fields such as robotics, by designing agents with a central processing unit that commands the actuators.
This simplicity, however, comes at a cost.
The most unremarkable actions that animals perform with little cognitive load, such as grasping an object or locomotion, have proved to be painstakingly difficult to implement in robots~\cite{Pfeifer2006Book}.
For decades now, an alternative approach has been proposed that has improved the performance of robotic agents~\cite{Brooks1991AI}.
\par
Since then, embodiment, has enabled engineering of more robust and adaptable robots, inspired from biological organisms.
\citeauthor{Pfeifer2007Sci} present insect locomotion as a very convincing example of taking advantage of embodiment principles in robotics.
Insects demonstrate coordinated walking and running, which given their six legs, pose a challenging problem with dozens of degrees of freedom, in particular on uneven terrains.
It is plausible to assume they do not solve the inverse kinematic problem for all their joints at all the times, which was the classic approach in robotics and requires enormous computational resources.
However, by taking embodiment into account, pushing back a single leg, which could be detected by angle sensors in the joint, could command all the other joints to move in the ``correct" direction.
This way, a low level communication between the legs could be exploited to achieve leg coordination without any central controller in the nervous system~\cite{Pfeifer2007Sci}.
\par
In the animal kingdom, embodiment enables both cognition--even the most abstract processes, like mathematical reasoning~\cite{Lakoff2000Book}-- and action.
In this framework, behavior is not reduced to internal computations, rather it is the manifestation of intricate brain-body-environment interactions.
Perception of the external world relies upon how the information is channeled through different parts of the body and differences in the shape of body parts alters the incoming and outgoing signals~\cite{Gomez2019Neuron}.
The body also shapes the way we interact with our environment and subjects us to the laws of physics---having to deal with gravity, friction, and forward arrow of time~\cite{Pfeifer2006Book}.






% There is convincing experimental evidence that our perception of time is not isomorphic with the time measured by human-made clocks and strongly depends on contextual factors affecting the subjects' attentional, motivational or emotional state~\cite{Paton2016Sci, Coull2004Sci, Droit2007TICS, Effron2006Emotion, Pariyadath2007Plos, Gable2012PsychSci}.\footnotemark
% \footnotetext{
%     Time flies when we are fully engaged in an activity (giving a 20 minutes-long neuroscience talk) but can terribly drag when we are bored (listening to a 20 minutes-long neuroscience talk).
%     }
% Both emergent and dedicated algorithms for time estimation account for the state-dependant variance in time perception by a neuromodulator-based slowing or speeding of neuronal clocks~\cite{Paton2016Sci, Simen2016Sci}.
% Alternatively, the strong influence of contextual factors on time perception may indicate that mechanisms beyond internal neural activity play a primary role in how humans and animals perceive time and adapt their behavior accordingly.
% Indeed, the embodiment framework proposes that important aspects of cognition are determined by physical features of the body, which in turn influence how humans or animals interact with their environment~\cite{Pfeifer2006Book}.
% That is to say, animals (and humans) may primarily replace the abstract concept of having to estimate a certain duration with an embodied strategy that happens to last the same duration.\footnotemark
% \footnotetext{
%     {\color[rgb]{1,0,0} SAY SOMETHING ABOUT EMBODIMENT FROM THE BOOK...}
%     }
% In tasks whereby a fixed time interval must be respected to obtain a reward, animals do not stay immobile, rather they take advantage of the structure of their environment to develop ritualistic motor routines whose durations match these task-specific temporal constraints~\cite{Gouvea2014, Kawai2015, Rueda2015NN}.
% Although humans can seemingly estimate time, either by counting in their heads or otherwise, it is difficult to separate this mental faculty from memories of counting the passing seconds aloud, which in turn is a sequence of coordinated movements across respiratory, laryngeal and supraglottal articulatory systems, or in general, any motor memory~\cite{Coull2018}.
% In addition, humans naturally tend to estimate time intervals through rhythmical movements of some of their body parts (by flicking fingers, moving arms, nodding heads or tapping feet).\footnotemark
% \footnotetext{
%     It is noteworthy that, to reliably estimate time, humans have created clocks, that indicate time by moving objects in space and extensively use metaphors containing movement and space references when speaking of time (\emph{holidays are approaching}, \emph{time flies})~\cite{Nunez2013TICS,Winter2015Cortex}.
%     }
% It has also been recently argued that explicit perception of time may be constructed implicitly through association between the duration of an interval and its sensorimotor content~\cite{Coull2018}.
% Thus, animals (and humans) may primarily rely on the movement of their body in space to estimate time~\cite{VoletRev2013}, which incidentally explains why brain regions involved in movement control, such as the \gls{bg}, have consistently been associated with time perception~\cite{Coull2018}. 






\subsection{Embodied Clock}
\label{ch:intro:Embodiment:Embodied Clock}
\epigraph{Time by itself does not exist\ldots It must not be claimed that anyone can sense time apart from the movement of things.}
{\textit{Lucretius, Book 1}}
\noindent




\section{Basal Ganglia as a Clock}
\label{ch:intro:BGTime}
Many brain structures have been proposed to contribute to time estimation.
The \gls{bg}, a set of interconnected subcortical nuclei, are especially of interest, since they are directly involved in motor processes as well~\cite{Grillner2015}.
Moreover, the \gls{bg} are also involved in reinforcement learning---selecting actions in an uncertain world in a way that maximizes reward in the long term~\cite{Petter2018}.
Such learning necessitates an understanding of temporal contingencies in order to maximize some future reward.
Behavioral data also supports that animals build probabilistic models for timing~\cite{li2013PNAS}.
In general, execution of any complex behavior requires proper timing of the comprising sub-actions.
\par
The \gls{bg} are often implicated in timescales of several hundreds of milliseconds to several seconds~\cite{Paton2018NeuronRev}.
Evidence of involvement of the \gls{bg} in timing stems from a variety of sources, including pathologies such as \gls{pd}, lesion studies, and pharmacological and genetic manipulations.
\par
Following the taxonomy discussed in \autoref{ch:intro:taxonomy}, there is some evidence of involvement of the \gls{bg} in sensory timing.
\Citeauthor*{Rao2001} reported encoding of time intervals in the human striatum in a task in which subjects reported whether an interval were shorter or longer than a standard interval of 1200~ms.\footnotemark\
They also observed a dynamic network of cortical activity in inferior parietal, premotor, and dorsolateral prefrontal cortex.
These nodes in the network were attributed to different components of temporal processing, respectively, attention, memory, and interval comparison.
They collectively concluded implication of the ``striatal dopaminergic neurotransmission in hypothetical internal timekeeping mechanisms"~\cite{Rao2001}.
\footnotetext{
    This paradigm is commonly referred to as ``interval categorization task".
    }
\Citeauthor*{Pouthas2005} also investigated interval categorization for two durations (450~ms and 1300~ms).
They observed ramping striatal activity during both intervals.
They concluded a direct role of the basal ganglia in duration estimation, and that the caudate nucleus ``may support a clock mechanism"~\cite{Pouthas2005}.
Similar evidence exist in other species too.
\Citeauthor*{Gouvea2015Elife} trained rats in a sensory categorization task to judge whether an interval is shorter or longer than 1.5~s.
They decoded animals' choice and elapsed time from ensembles of striatal neuronal activity, whereas apparent behavior in an overhead video failed to do so.
Transient inactivation of the \gls{ds} impairs performance, however, it doesn't cause a systematic under-- or over--estimation~\cite{Gouvea2015Elife}.
\par
% Furthermore, the \gls{bg} are also well studied for their role in motor timing.
% In a range of disorders affecting the \gls{bg}, including \gls{pd}, Huntington's disease, Tourette's syndrome, drug abuse, and attention deficit disorder, an altered perception of time, or temporal patterns have been reported~\cite[see][]{Paton2018NeuronRev}.
Furthermore, the \gls{bg} are also well studied for their role in motor timing.
\Citeauthor*{Matell2003} trained rats to receive a reward in a fixed interval reinforcement schedule.\footnotemark\
The interval alternated between 10~s (25\% of trials) and 40~s (75\% of trials).
After learning, animals increased their lever press rate around the reinforced intervals.
Electrophysiological recordings from the striatum show neurons with tuned firing rate only around 10~s interval, but not 40~s, while apparent behavior of the animals is similar.
The authors then suggest that a population of duration-coding cells, each tune to different values, could accurately represent the elapsed time~\cite{Matell2003}.
\footnotetext{
    In operant conditioning, fixed interval reinforcement schedule refers to a type of conditioning whereby a response is reinforced (i.e., rewarded) only if a certain period of time (i.e., interval) has elapsed.
    }
\Citeauthor*{Mello2015} also used a similar task for intervals ranging between 12~s to 60~s.
They found striatal cells that rescaled their activity when intervals changed.
As rats adjusted to the new interval, time estimations decoded form population dynamics predicted animals' timing performance.
In another study, \Citeauthor*{Bakhurin2017JNeuro} used a conditioning paradigm to signal delayed reward delivery (2.5~s after cue onset).
Individual neurons recorded in the striatum and orbitofrontal cortex display sequential activity during the interval.
A machine learning algorithm was then trained to decode the elapsed time from the stimulus onset.
They show that both striatal and cortical networks ``encoded time, but the striatal network outperformed the orbitofrontal cortex".
Interestingly, removing the neurons modulated by licking activity from the decoder significantly reduced its performance, however, it still remained higher than chance level~\cite{Bakhurin2017JNeuro}.
\par
Another source of impact in the \glsentrylong{bg} is the neuromodulatory effect of \gls{da}.
\Glsentrylong{da}'s role in reward processing and circuit dynamics of the striatum will be discussed later in sections~\ref{intro:BGMotor} and \ref{intro:BGAnatomy}.
However, \gls{da} is also believed to be involved in timing~\cite{Paton2018NeuronRev}.
In a peak interval procedure\footnotemark, \Citeauthor*{DeCorte2019} found that \gls{d2} blockade delayed start and stop times for an interval of 6~s.
Whereas, blockade of \glspl{d1} delayed stop times.
Then they stress the role of the \gls{ds} in timing, with \gls{da} ``being particularly critical for the temporal control of action"~\cite{DeCorte2019}.
\footnotetext{
    Peak interval procedure is a common task used to study timing.
    Similar to fixed interval schedules, a cue indicates that a response will be reinforced only after a certain period of time has elapsed.
    The profile of the response around the interval is then studied.
    }
\Glsentrylong{da} neurons encode reward prediction errors which requires accurate reward predictions~\cite[see][]{Berke2018NN}.
\Citeauthor*{Takahashi2016} recorded from \gls{da} neurons of rats while they performed a task with uncertainty in reward timing and reward number.
Neuronal activity showed error signals in response to both types of prediction error, however, after ventral striatal lesions, neurons only responded to changes in reward number, and not reward timing.
These results suggest that time-dependant component of reward prediction of \gls{da} neurons might rely on the ventral striatum~\cite{Takahashi2016}.
In an interesting study, \Citeauthor*{Paton2016Sci} measured and manipulated the activity of \gls{da} neurons in a 1.5~s interval categorization task.
\Gls{da}ergic activity predicted animal's time estimates.
Transient activation/inhibition of \gls{da} neurons caused under-/over-estimation of the interval.
Hence, they concluded that ``\gls{da} neurons, which are so central to reward processing, exert control over time estimation"~\cite{Paton2016Sci}, although these results reflect \gls{da} function in general, not specifically in the \gls{bg}.
Similar to scaling of neuronal activity in the striatum~\cite{Mello2015}, \gls{da} concentration in the \gls{ds} is also scalable to time intervals in several second time range~\cite{Howard2017}.
However, \citeauthor{Howard2017} then conducted a series of experiments and concluded that the \gls{da} signal in the \gls{ds} does not reflect interval timing \textit{per se}, rather it is specific to behavioral choice of action~\cite{Howard2017}.