\section[Motivation, Question and More]{Motivation, Question and the Organization of the Thesis}
\label{intro:question}

The work presented in this thesis has two fronts that seem unrelated, but in this chapter I tried to present them on a conceptual continuum.
First part is concerned with the question of how animals often act as though they have a sense of time.
Enormous body of experimental and theoretical research implicates plenty of brain areas as providers of a time signal.
Such a mechanism could be affected by external factors (e.g., reward rate and motivation), however, it is usually assumed to be the means by which well-timed actions are generated.
This is what I call ``internal time estimation'', not that the world exterior to the brain is irrelevant, but meaning that the brain has a sense of time on its own that underlies behavior.
This mechanism is appealingly simple, predictive of many behavioral phenomena, and backed by neurophysiological data.
Alternatively, we hypothesized that there is no sense of time per se, and that time is perceived through interactions with the environment.
In other words, the duration of an interval is displaced by its sensorimotor content.
Since movement is among the most basic functions of the nervous system and inevitably, it takes a certain duration to execute any action, elapsed time could just be inferred from actions (or similarly, sensory processes).
Such an ``embodied time estimation'' provides a much more parsimonious explanation, and is in alignment with the long-reported and replicated observation across many species that animals produce stereotyped motor sequences under temporal constraints.
Nonetheless, this hypothesis has not been very popular!
Perhaps partly due to technological limitations to monitor a wide range of animal behavior (in rodents, from locomotion to whisking and sniffing), especially in standard experimental paradigms inside Skinner boxes; and in my opinion, partly due to a general brain-centric view where the brain is the puppeteer of the body.
\par
To test this hypothesis, I used a novel behavioral paradigm developed by~\citeauthor{Rueda2015NN} that is a powered treadmill with a reward contingent on timing of appetitive approaches (details are discussed in \autoref{ch:methods:exp}).
This task allows monitoring of location of the animals (and kinematics of their locomotion).
Powered treadmill enabled us to manipulate dynamics of the environment in order to facilitate or hinder exploitation of the stereotyped motor sequences that we hypothesized are essential for solving the task.
We assumed if timing was internally-driven, animals should be able to perform the task without resorting to the stereotyped motor strategies.
Results from these experiments are presented in \autoref{ch:time}.
\par
Second facet of this work deals with the problem of implementation, i.e., how the brain generates the motor 
sequence it presumably uses to keep track of time.
Classic models of the \gls{bg} implicate the \gls{dms} in early phases of learning, and the \gls{dls} in executing the learned sequences, or controlling their kinematics.
Results from earlier work in the lab suggest that the overall behavior of the animals following transient inactivation of the \gls{dls} remains intact, although more variable~\cite{Rueda2015NN}.
Thus, in this work, using a similar approach, we aimed to specify the function of the striatum in development and execution of this behavior.
In particular, I evaluated the role of the striatum, the main input to the \gls{bg}, in learning and controlling the kinematics of a motor sequence, by permanently lesioning its subareas (details are discussed in \autoref{ch:methods:tech}) in both na\"ive and trained animals.
Results from these experiments are presented in \autoref{ch:lesion}.
\par
Finally, I synthesize an overall view of my Ph.D.\ project and discuss its meaning and implications, as well as some of the shortcomings and directions for future works in \autoref{ch:discussion}.