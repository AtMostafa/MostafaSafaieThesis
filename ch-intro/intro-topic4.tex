\section[Motivation, Question and More]{Motivation, Question and the Organization of the Thesis}
\label{intro:question}

The work presented in this thesis has two fronts that although might seem unrelated at the first glance, are conceptually connected.
First is concerned with the question of how animals often act as though they have a sense of time.
Enormous body of experimental and theoretical research implicates plenty of brain areas as providers of a time signal.
Such a mechanism could be affected by external factors (e.g., reward rate and motivation), however, it is usually assumed to be the means by which well-timed actions are generated.
This is what I call ``internal time estimation", not that the world exterior to the brain is irrelevant, but meaning that the brain has a sense of time that underlies behavior.
Alternatively, we hypothesized that there is no sense of time per se, and that time is perceived through interactions with the environment.
In other words, the duration of an interval is displaced by its sensorimotor content.
Such ``embodied time estimation" provides a much more parsimonious explanation, and is in alignment with the long-reported and replicated observation across many species that animals produce stereotyped motor sequences under temporal constraints.
Nonetheless, this hypothesis has not been very popular!
Perhaps partly due to technological limitations to monitor a wide range of animal behavior (in rodents, from locomotion to whisking and sniffing), especially in standard experimental paradigms inside Skinner boxes; and in my opinion, partly due to a general brain-centric view where the brain is the puppeteer of the body.
\par
Here, I used a novel behavioral paradigm developed by~\citeauthor{Rueda2015NN} i.e., a powered treadmill with reward contingent on timing of appetitive approaches (details are discussed in \autoref{ch:methods:methods}).
This task allows monitoring of location of the animals and kinematics of their locomotion.
Powered treadmill enabled us to manipulate dynamics of the environment in order to facilitate or hinder exploiting stereotyped motor sequences that we hypothesized are essential for solving the task.
Results are presented in \autoref{ch:time}.
\par
Second facet of this work deals with the problem of implementation, i.e., how the brain generates the motor sequence it presumably uses to keep track of time.
In particular, I evaluated the role of the \gls{ds}, the main input to the \gls{bg}, in learning and controlling the kinematics of the motor sequence by permanently lesioning its subareas (details are discussed in \autoref{ch:methods:methods}).
Classic models of the \gls{bg} implicate the \gls{dms} in early phases of learning, and the \gls{dls} in executing the learned sequences.
Results from these experiments are presented in \autoref{ch:lesion}.
Finally, in \autoref{ch:discussion} I synthesize an overall view of my Ph.D. project and discuss its meaning and position in the literature.