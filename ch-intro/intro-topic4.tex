\section[Motivation and Organization]{Motivation, and the Organization of the Thesis}
\label{intro:question}

The work presented in this thesis has two fronts that although might seem unrelated at the first glance, are conceptually connected.
First is concerned with the question of how animals often act as though they have a sense of time.
Enormous body of experimental and theoretical research implicates plenty of brain areas as providers of a time signal.
Such a mechanism could be affected by external factors (e.g., reward rate and motivation), however, it is usually assumed to be the means by which well-timed actions are generated.
This is what I call ``internal time estimation", not that the world exterior to the brain is irrelevant, but meaning that the brain has a sense of time that underlies behavior.
Alternatively, we hypothesized that there is no sense of time per se, and that time is perceived through interactions with the environment.
In other words, the duration of an interval is displaced by its sensorimotor content.
Such ``embodied time estimation" provides a much more parsimonious explanation, and is in alignment with the long-reported and replicated observation across many species that animals produce stereotyped motor sequences under temporal constraints.
