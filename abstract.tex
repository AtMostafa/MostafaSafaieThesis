The question of how animals adapt their movements to take advantage of temporal regularities in their environment is still not well-understood, especially in the suprasecond timescale.
It has been proposed that time estimation is internally-driven, using either a central neuronal clock, or emergent self-sustained dynamics across ensembles of neurons.
Alternatively, animals may use embodied strategies, such as a motor routine whose execution takes the same duration as the interval they need to estimate.
Implementation of either timing mechanism in the brain is a debated issue, however the dorsal striatum (DS) is reported to be implicated by both.
The DS neurons are reported to represent elapsed time, and perturbation of DS activity affects temporal perception.
On the other hand, the DS is a known motor area, thought to be involved in selection/repression of purposive actions, driving their execution on a moment-to-moment basis, or modulating their speed.
Here, we used a task in which rats freely moving on a powered treadmill could obtain a reward if they approached it after a fixed interval.
Most animals took advantage of the treadmill length, speed, and its moving direction, and by trial-and-error, developed a wait-and-run motor routine, the execution of which resulted in precise timing of their reward approaches.
We then addressed two questions:
    whether animals are able to time their behavior without resorting to this motor routine;
    and how the DS contributes to the performance of this motor routine.

\par

To address the first question, we trained na\"ive animals in modified versions of the task, specifically designed to hamper the development of this motor strategy.
Compared to rats trained under the normal protocol, these animals never reached a comparable level of timing accuracy.
We conclude that motor timing critically depends on the ability of animals to develop motor routines adapted to the structure of their environment.

\par

Secondly, the exact contribution of the DS to the execution of such motor routines is unclear.
Unexpectedly, following DS lesions, the performance of the motor routine was spared, but altered in peculiar ways:
    animals reduced their running speed and waiting period of their routine.
Complementary experiments demonstrated that DS lesions did not affect the animals' motivation, their ability to perform motor routines or to control their running speed.
We conclude that lesions of the DS increased the sensitivity to energy expenditure.
Thus, we propose that the DS computes an effort signal that modulates the kinematics of purposive actions.