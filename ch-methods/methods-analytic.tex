\section{Analytical Tools} \label{ch:methods:analytics}

\subsection{Statistical Analysis}
All statistical comparisons were performed using a permutation test previously described in~\cite{Fujisawa2008NN}.
This non-parametric method alleviates many concerns in statistical hypothesis tests, such as distribution assumptions (e.g., normality assumption under analysis of variance), error inflation due to multiple comparisons, and sensitivity to unbalanced group size.
\par
To simplify the description, let's assume, we have ${\mathbf{X}=[X_1, X_2,...,X_n]}$, where $X_i$ is the set of $ET$s of session~$i$.
Similarly, we have $\mathbf{Y}$ that contains $ET$s of all the sessions from another condition.
Here, the null hypothesis states that the assignment of each data point in $X_i$ and $Y_i$ to either $\mathbf{X}$ or $\mathbf{Y}$ is random, hence there is no difference between $\mathbf{X}$ and $\mathbf{Y}$.
\par
In short, the test statistic was defined as the difference between smoothed (using Gaussian kernel with $\sigma =0.05$) $\mathbf{X}$ and $\mathbf{Y}$ for each session~$i$: $D_0(i)$.
At this point, we generated one set of surrogate data by assigning each $ET$ of session $i$ to either $X_i$ or $Y_i$, randomly.
For each set of surrogate data, the test statistic was calculated, i.e.,~$D_m(i)$.
This process was repeated 10,000 times for all the statistical comparisons in this study, obtaining: $D_1(i),\ldots,D_{10000}(i)$.
\par
At this step, two-tailed pointwise p-values could be directly calculated for each $i$, from the $D_m(i)$ quantiles~\cite[see][]{Fujisawa2008NN}.
Moreover, to compensate for the issue of multiple comparisons, we defined global bands of significant differences along the session index dimension.
From 10,000 sets of surrogate data, band of the largest $\alpha$-percentile was constructed, such that less than 5\% of $D_m(i)$s broke the band at any given session $i$.
This band (denoted as the \textit{global band}) represents the threshold for significance, and any break-point by $D_0(i)$ at any $i$ is a point of significant difference between $\mathbf{X}$ and $\mathbf{Y}$.
\par
In cases of comparing only two sets of data points, the same algorithm was employed, having only one value for index $i$.
If none of the $D_m(i)$s exceeded $D_0(i)$, the value $p<0.0001$ was reported (i.e., less than one chance in 10,000).
