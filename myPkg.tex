\newcommand{\Autoref}[2]{\hyperref[#1]{\autoref{#1}#2}}
\let\subsectionautorefname\sectionautorefname
\let\subsubsectionautorefname\sectionautorefname

% Font Setting
\usepackage[utf8]{inputenc}
% \usepackage{newtxmath,newtxtext}
\usepackage[lining]{ebgaramond}

\usepackage[headheight=15pt, textheight=650pt, footskip=1.5cm, marginparwidth=0cm, marginparsep=0cm]{geometry}

\usepackage[
    backend=biber,
    style=numeric-comp,
    sorting=none,
    backref=true,
    maxcitenames=2
    ]{biblatex}
\addbibresource{ref.bib}
\DefineBibliographyStrings{english}{
    backrefpage = {cited on page},
    backrefpages = {cited on pages:}
    }

\usepackage{epigraph}
\setlength{\epigraphwidth}{0.7\textwidth}

\usepackage{xcolor}

\usepackage{enumitem}

\usepackage[symbol,bottom,hang,flushmargin]{footmisc}
\renewcommand{\thefootnote}{\roman{footnote}}

\usepackage[toc,nogroupskip]{glossaries}
\glsdisablehyper
\makeglossaries
\setacronymstyle{long-short}
\loadglsentries{abbrv}

\usepackage[font=small,labelfont=bf]{caption}

\usepackage[activate={true,nocompatibility},final,tracking=true,kerning=true]{microtype}

% \usepackage{fancyhdr} %added to the puthesis.cls
\fancyhead[L]{\nouppercase{\leftmark}}
\fancyhead[R]{}
\renewcommand{\headrulewidth}{1.5pt}

