\begin{figure}[bt!]
	\begin{center}
		\includegraphics[width=\textwidth]{ch-time/figures/CtrlTrd.pdf}
		\caption[Control Condition]
		{\textbf{Animals developed a unique stereotyped motor sequence.}
		\textbf{a)}
		\textit{Left}: trajectory of a representative animal in 9~consecutive trials of the 1st (\textit{top}) and 30th (\textit{bottom}) sessions.
		On the y-axis, 0~and 90~indicate the treadmills front and rear wall, respectively.
		\textit{Right}: trajectories for all trials of the corresponding sessions on the right.
		Distributions of initial positions for correct (green) and error (red) trials are shown on the y-axis.
		Black horizontal boxplots depict \gls{et} range (center line, median; box, 25th and 75th percentiles; whiskers, 5th and 95th percentiles).
		\textbf{b)}
		Median \gls{et}.
		Circles indicate group median and error bars, the median range (25th and 75th percentiles) across animals for \gls{et} and on the right y-axis, standard deviation of \gls{et} values.
		The magenta line shows the \gls{gt}.
		\textbf{c)}
		Median trajectory for the 1st (\textit{left}) and 30th (\textit{right}) training sessions.
		Each line represents a single animal ($n=54$).
		\textbf{d)}
		Percentage of trials in which animals performed the stereotyped routine.
		\textbf{e)}
		Probability distribution function~(PDF) of the position of the animals at the beginning of each correct (green) and error (red) trial, from sessions \#20 to \#30.
		Dashed lines represent cumulative distribution functions (right y-axis).
		The gray area indicates that in trained animals,~80\% of correct trials began with the animal located near the front of the treadmill.
		\textbf{f)}
		PDF of the maximum position along the treadmill.
		Only trials in which animals were initially located in the front of the treadmill (gray area in panel~e) were included.
		}
		\label{fig:time:CtrlTrd}
	\end{center}
  \end{figure}