\selectlanguage{french}
Comment les animaux adaptent leur comportement pour tirer profit des régularités temporelles de leur environnement est une question difficile, en particulier pour ce qui est des intervalles de l'ordre de quelques secondes.
Il a été proposé que l'estimation du temps est mesurée de manière interne, en utilisant soit une horloge neuronale, soit la dynamique émergente et auto-entretenue des ensembles de neurones.
Les animaux pourraient également utiliser des stratégies incarnées (`embodied'), telles que des routines motrices, dont l'exécution prend la même durée que l'intervalle qu'ils doivent estimer.
La validité relative de ces deux mécanismes n'est toujours pas établie.
De nombreuses régions du cerveau sont impliquées dans l'estimation du temps, dont l'une, le striatum dorsal (DS), présente un intérêt particulier.
En effet, les neurones du DS représenteraient le temps écoulé et la perturbation de l'activité du DS affecterait la perception temporelle.
D'autre part, le DS est une zone motrice connue, dont la fonction est également débattue (sélection/répression d'actions, génération des mouvements ou modulation de leur vitesse).
Ici, nous avons utilisé une tâche dans laquelle des rats se déplaçant librement sur un tapis roulant motorisé pouvaient obtenir une récompense s'ils s'approchaient de l'avant du tapis après un intervalle de temps fixe.
La plupart des animaux profitait de la longueur, de la vitesse et de la direction du tapis roulant et, par tâtonnement, développait une routine dont l'exécution permet de respecter la règle spatio-temporelle et d'obtenir une récompense.
Nous avons ensuite abordé deux questions :
    Les animaux sont-ils capables de de s'adapter à la règle spatio-temporelle sans avoir recours à cette routine motrice ?
    Comment la DS contribue-t-il à la performance de cette routine motrice.

\par

Pour répondre à la première question, nous avons entraîné des animaux dans des versions modifiées du test original, spécialement conçues pour empêcher le développement de leur routine motrice.
Par rapport aux rats entraînés selon le protocole original, ces animaux n'ont jamais atteint un niveau comparable de précision temporelle.
Nous en concluons que l'adaptation précise à une contrainte temporelle est facilité par la capacité des animaux à développer des routines motrices adaptées à la structure de leur environnement.

\par

Pour répondre à la deuxième question nous avons réalisé des lésions du DS.
De manière inattendue, à la suite de lésions du DS, l'exécution de la routine motrice a été épargnée, mais modifiée de manière particulière : 
    les animaux ont réduit leur vitesse de course et la période d'attente de leur routine.
Des expériences complémentaires ont démontré que les lésions du SD n'affectaient pas la motivation des animaux ni leur capacité à effectuer des routines motrices ou à contrôler leur vitesse de course.
En nous appuyant sur des modélisations du comportement, nous concluons que les lésions du DS augmentent la sensibilité des animaux à la dépense énergétique.
Ainsi, nous proposons que le DS calcule un signal d'effort qui module la cinématique des actions intentionnelles.
\selectlanguage{english}