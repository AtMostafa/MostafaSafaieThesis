\section{On the Other Hand}
\epigraph{The idea that the future is unpredictable is undermined every day by the ease with which the past is explained.}
{\textit{Daneil Kahneman, Thinking, Fast and Slow}}
\noindent
It is time to discuss some issues from which I think my thesis project suffers, some flaws that might result in interpretational limitations.
Firstly, the treadmill task does not require a clear and distinct operant response, rather the animal crosses the infrared beam, the location of which is unmarked and unknown to the animal.
This is uncharacteristic for a time-estimation task.
Retrospectively, had we installed a simple lever or nose-poke above the reward port to register the entrance times, it would have provided a more straightforward timing task.


clear motor response
not favoring the motor
refutability

Reward profile in lesion
Redundancy due to lesion
small effect of max pos
smaller DMS lesions and the ventricule
variability in strategies(not doing the routine)






\section{Future Work}

opto
DA manipulation
infinite treadmill

Still, why animals and humans seem to favor embodied and interactive timing strategies over purely internal mechanisms is not clear.
Insights regarding this question might be obtained by considering adaptive behavior in an evolutionary perspective\cite{Cisek2019} and time in the context of ecologically valid timing tasks\cite{vanRijn2018}.


Future work should further investigate how to mathematically capture the algorithms underlying temporally-constrained actions and decisions and the possible contribution of the striatum to processing multimodal (sensorimotor, cognitive, emotional) contextual information together with dopamine-mediated feedback and motivational signals.


Future studies should investigate whether signaling effort and urgency are the two sides of a unique function implemented in the \gls{bg} to maximize the reward rate while minimizing costs.




why embodied is preferred?