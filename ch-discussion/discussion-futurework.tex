\section{On the Other Hand}
{\singlespacing \epigraph{The idea that the future is unpredictable is undermined every day by the ease with which the past is explained.}
{\textit{Daneil Kahneman, Thinking, Fast and Slow}}}
\noindent
It is time to discuss weak points of this project, some possible flaws that might result in interpretive limitations.
First issue relates to the structure of the task.
The treadmill task does not require a clear and distinct operant response, rather the animal crosses the infrared beam, the location of which is unmarked and unknown to the animal.
This is uncharacteristic for a time-estimation task.
In retrospect, had we installed a simple lever or nose-poke above the reward port to register the entrance times, it would have provided a more straightforward timing task.
The task was originally designed to be used at faster speeds (30~cm/s) and such a mechanism was perhaps deemed impractical.
%======
Next point is also with regards to the task.
In its nature, I think, the treadmill task favors a motor strategy, and this might have biased our results.
Aside from the immobile condition, the treadmill always moves and so imposes a dynamic environment to the agent, an environment in which avoiding motor activity is not even possible, since eventually the animal would arrive to the back wall and \textit{has to} move.
The immobile condition was an attempt to remedy this problem.
Indeed, we showed that better performance was correlated with more movement along the treadmill and visually, we observed that our best-performing animal developed a stereotyped ritual in interaction with the back wall of the treadmill.
However, even in this condition, animals couldn't simply stay in the reward area, and presumably \textit{estimate} the goal time duration, at least they had to move a few centimeters backward to prevent premature interruption of the beam.
Ideally, to test the embodied view, the time-estimation task should not allow any motor activity, perhaps by penalizing movements or incentivizing immobility.
However, a new problem arises:
    what would be the ecological relevance of such a task, especially in suprasecond timescale?
And this is closely linked to my last remark.
Even though learning about embodiment and its implications immensely influenced my thinking, I fear that in the context of timing, it might be unfalsifiable.
For instance, even if human subjects are asked to not move at all during a time-estimation task, and they perfectly follow the protocol, one could argue (and I do argue) that they still mentally picture a motor activity or a moving object~\cite[also][]{Coull2018}.
Overall, the best argument in favor of embodied time-estimation is its parsimony, and that adopting this perspective affords a great explanatory power.
\par
The lesion experiments are also not perfect.
One issue is that the quantity of reward progressively decreased for correct trials with larger entrance times (8~s vs.\ 10~s, \autoref{fig:methods:taskRules}).
Thus, had lesioned animals started running at a position similar to normal rats using a slower speed, they would have received a slightly smaller drop of reward.
The most skeptical reader might suggest that this is the reason why animals tend to wait less, to avoid smaller rewards.
However, the reward magnitude drops at a low rate, therefore I do not think a $\sim$7\% smaller reward is noticeable.
In any case, at least for the lesion experiments, a constant reward size after the goal time would have been preferred.
%======
Lastly, one issue that might have downgraded our effect sizes, especially with regards to the maximum position analysis, was the behavioral variability.
Some animals developed different strategies to solve the task (\autoref{fig:appendix:BadCtrl}).
At the time, to respect the diversity of natural behavior, we indiscriminately carried on with the lesion experiments for all the animals.
Later on, while processing the data, I realized most outlier data points belonged to animals with strange pre-lesion behavior, but at this stage we did not have any excuse to exclude those animals, and we did not.
Instead, we explicitly set criteria for including animals in the maximum position analysis (\autoref{fig:lesion:maxPos}).
These criteria might seem arbitrary to some, although they served the purpose of the analysis and were set blind to the performance of individual animals.
I suppose performing the lesion only in animals that behaviorally conformed to the wait-and-run routine would have rid us of much of the \textit{extra} variability and also would have been scientifically justified.


\section{Future Work}
Thus far, I presented, hopefully, convincing evidence in support of the necessity of motor routines for accurate timing and that the \gls{ds} sets the sensitivity to the expended effort in motor routines.
Then some limitations in the design of the task and the interpretation of our data were discussed and now I propose some directions for future research.
\par
% We showed that lack of a simple motor strategy leads to the deterioration of timing performance.
% Therefore, if one considers a continuum for timing strategies on an embodied--internal axis, then, why the internal mechanism is less accurate than the embodied one?
% In the context of this work, I think the related question is: why 
We showed that the timing performance deteriorates in the absence of an \textit{easy} motor strategy, i.e., a motor routine that is adapted to the environment, like the wait-and-run routine in the control treadmill task.
The worst performances were observed in the versions of the task wherein stereotyped execution of a 7~s long motor routine was arguably the hardest, like the immobile condition.
I would suggest that this is due to the inherent difficulty of reliably executing a suprasecond motor routine in a static environment, while in the control condition, the environment provides a reliable and salient cue, i.e., touching the back wall.
It remains speculative whether in the immobile condition (or any other condition) animals developed a different idiosyncratic routine, either postural or with their limbs, that we couldn't capture with our position-tracking system.
Better behavioral quantification, possible with the advent of modern technologies such as \textit{DeepLabCut}~\cite{DeepLabCut2018NN}, in ecologically-valid timing tasks~\cite{VanRijn2018TICS} could detect such previously-unnoticed routines.
Furthermore, it also requires future investigations into theoretical reasons why inferring time from internal dynamics of neural networks seemingly results in higher variability than relying on physical interactions with the environment.
\par
In the second part, we used permanent fiber-sparing lesions to study the role of the striatum in development, execution, and control of the wait-and-run motor routine.
Lesion is a useful tool to establish a causal `instructive' function for the striatum~\cite{Otchy2015Nature}, however it does not provide any information as to how direct/indirect pathways are involved.
Recent genetic tools allow pathway-specific perturbation of neural activity in rats~\cite{Pettibone2019eNeuro}, which is absolutely compatible with our task and setup and would complement our conclusions tremendously.
I speculate that direct (\textit{indirect}) pathway stimulation would cause under- (\textit{over-}) sensitivity to cost, somewhat contrary (\textit{similar}) to what we observed by lesions.
Moreover, dopamine manipulation would also further this work's proposition.
It has been suggested that dopamine in the striatum carries a `motor motivation' signal, and its disruption in Parkinson's disease leads to a more conservative energy-expending policy~\cite{Mazzoni2007}.
Therefore, I suppose that animal models of Parkinson's, dopamine-depleted either by 6-hydroxydopamine lesions or progressive degeneration of dopaminergic neurons~\cite{Panigrahi2015Cell}, would show similar results to our striatal lesioned animals.
Similarly, transient dopamine activity manipulation using optogenetics could be another future direction.
However, the treadmill task does not allow for clear behavioral predictions, since both prokinetic behavior due to the up-regulated dopamine, and energy-efficient behavior due to dopamine paucity may seem identical:
remaining near the reward port.
Finally, I reiterate that the theories of \textit{urgency} in decision-making and \textit{effort} in motor control might indeed reflect a common underlying function of the basal ganglia to maximize the capture rate (\autoref{ch:intro:cost}).
Direct assessment of this postulate could be possible in a task wherein the agent reports a decision (e.g., by a nose poke) via a discernible motor output (e.g., by running toward the nose port).
For instance, consider a T-maze task with a sensory cue at the base of the central stem determining the arm in which the reward will be delivered~\cite[i.e., a combination of][]{Zuo2019CurrBiol,Barnes2005Nature}.
Such a setup would illustrate the urgency in decision making with the time the animal takes to commit~\cite{Thura2017Neruon}, and the effort in motor control with the velocity with which it approaches the reward port.
Manipulating the neural activity could then reveal how urgency and effort are related.
Future studies are needed to delineate this possibility and the role of the dopamine in the effort framework.