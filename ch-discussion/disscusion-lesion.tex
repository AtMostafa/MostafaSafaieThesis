\section{Striatal Function}
\label{ch:disscusion:lesion}

The striatum can powerfully influence the production of purposive movements.
Indeed, it is well-known that striatal dysfunction is the primary cause of the motor impairments (akinesia, bradykinesia, levodopa-induced dyskinesia) seen in \gls{pd}~\cite{Mink1996}.
In addition, activation of striatal \glspl{msn} forming the direct (indirect) \gls{bg} pathway facilitates (prevents) movement production through disinhibition (inhibition) of brainstem and forebrain motor regions~\cite{Kravitz2010Nature}.
This fundamental feature of the \gls{bg}'s functional anatomy combined with recordings and perturbations of striatal activity in various behavioral tasks, has led to three prevailing hypotheses regarding how the striatum contributes to the control of purposive movements: %#TODO: define purposive movement in a footnote
\begin{itemize}[noitemsep]
    \item the selection/repression of actions~\cite{Barnes2005Nature, Cui2013Nature, Klaus2017Neuron, Markowitz2018Cell};
    \item the moment-to-moment generation of overlearned motor sequences~\cite{Dhawale2019};
    \item and the modulation of movement speed~\cite{Kim2014EJN, Rueda2015NN, Barbera2016Neuron, Yttri2016Nature}.
\end{itemize}
The validity of these hypotheses is debated~\cite[for instance, see][]{Dudman2016CurrOpinNeurobiol} and, interestingly, they face common shortcomings.
For example, they fall short of explaining why lesioning or inactivating striatum's anatomical targets (i.e., the \gls{bg} output nuclei) in non-human primates only marginally alters the execution and speed-profile of overlearned motor sequences~\cite{Desmurget2010JNeurosci}; it even alleviates motor deficits observed in \gls{pd} patients~\cite{Turner2010CurrOpinNeurobiol}.
Moreover, in behavioral tasks typically used to probe the striatal motor function through perturbation of neuronal activity, it is next to impossible to disentangle whether failure to perform is due to inability to implement a decision into movement, or due to an impairment in higher-level processes (e.g., sensory processing and decision making), despite functional motor systems.\footnotemark
\footnotetext{
    This is referred to as the \emph{performance confound}.
    I only understood this concept after my supervisor came up with this rather bitter example:
    \textit{Imagine you cut up someone's legs and then say that they \underline{cannot} learn to run.}
}
\par
In this part of the study, to understand how the striatum contributes to the control of purposive actions, while limiting the performance confound as much as possible, we took advantage of the behavior displayed by animals in the treadmill task, identical to the one used in the time experiments under the `normal' condition (\Autoref{fig:methods:taskRules}{a}, and \Autoref{fig:lesion:task}{a}).
We trained a group of animals, including the same animals presented in \autoref{fig:time:CtrlTrd}.
They mostly developed the wait-and-run routine on the treadmill (but also see \autoref{fig:appendix:BadCtrl}).
After their performance on the task plateaued, they were randomly assigned to striatal lesion in different areas:
    \gls{dls}, \gls{dms}, or the entire \gls{ds}.
Excitotoxic lesions compared to lesions induced by electrical currents, have the benefit of sparing the passing fibers.
Moreover, due to their permanent nature, they are probably a more direct way to assess the function of the manipulated area~\cite{Otchy2015Nature}.
Following the lesion, animals were allowed to recover for $\sim$2~weeks.
After this recovery period, visually, they had normal behavior in their homecage.
Then, they resumed the training with identical task parameters to the sessions prior to the lesion.
After the striatal lesion, most animals could still perform in the task, with comparable proficiency to that of the pre-lesion sessions.
The most striking effect was a marked reduction in the running speed toward the reward (hence, a purposive action).
This slowdown was irreversible and well-correlated with the size of the lesion.
Interestingly, animals with reduced speed, also started to run earlier toward the reward, i.e., the maximum position of their trajectories were smaller.
This trait was also well-correlated with the lesion size, but not trivial to explain using the aforementioned models of the striatal function (more on this later).
\par
Noticeably, in a number of animals, arguably the ones with relatively larger lesions, executing the routine was impaired for the first few sessions ($\sim$5), and recovered afterward ($\sim$session~$+8$ onward).
They mostly stayed in front of the infrared beam and committed many error trials.
It could be argued that the lesion stopped performance of the motor routine, and especially animals with larger lesions stayed in the front since it is associated with the reward, evident by the strong preference for the reward area in both na\"{i}ve and trained rats (\autoref{fig:appendix:initPos}).
We designed a variant of the task to directly evaluate this argument.
In this `reverse' treadmill task, the conveyor belt moved toward the reward area.
Animals could just move to the back of the treadmill during the intertrial (while the treadmill motor is off, this is just regular locomotion which was not affected in lesioned animals) and upon trial start, stay still while the treadmill carried them to the reward area at the right time (\gls{et}~$=$~\gls{gt}).
Importantly, impaired execution of this ``run-and-wait'' motor routine could be clearly observed, because the animals should similarly stay in the front.
However, experimental data showed otherwise, animals kept performing the routine with no sign of deficiency.
These results demonstrate the spared ability of animals to perform motor routines.
\par
Another possible explanation of our data may be lack of more general motor control abilities, such as modulation of speed.





Theoretically, optimal control models indicate that remaining very close to the reward area minimizes energy expenditure (effort) by eliminating the fast speeds required to cross the treadmill~\cite{JuradoParras2020}.


Across post-lesion sessions, the same rats progressively waited longer before running toward the reward area, which allowed them to recover task proficiency (Fig. 1, C to F).
This suggested that the animals' ability to progressively wait longer before running toward the reward area was spared by the dS lesion.
Accordingly, we found that dS lesions performed before training did not compromise the rats' ability to learn the wait-and-run routine (fig. S7) but did reduce their speed, an effect that was also correlated with lesion size (fig. S4). 
\par
Overall, our results indicate that rats with dS lesions did not display fundamental impairment in motor control or action selection but behaved in a way that is most parsimoniously  explained by a higher sensitivity to effort with preserved motivation.
Indeed, after dS lesion, animals kept arriving on time in the reward area, but they started to run earlier (i.e., on a more intermediate portion of the treadmill) and at a slower speed.
Metaphorically speaking, the same effect would have been expected had we forced non-lesioned rats to perform the task with extra weight on their back. 
Thus, our work suggests that the dS contributes to the generation of an effort signal that influences the kinematic parameters of purposive actions.
Such a function is in line with the hypothesis that the dopamine projection to the dS provides a signal for implicit motor motivation (or global effort sensitivity), which in turn influences the vigor of most of the goal-directed movements performed human or animal subjects \cite{Mazzoni2007JN, Treadway2012JN, Reppert2018JNPhys}.
But, what would be the specific contribution of dS neurons to such function?
A possible answer is that dS neurons integrate motivational- and feedback-related dopaminergic signals with context- and action-specific information derived from their massive cortical and thalamic inputs  \cite{Hunnicutt2016Elife, Hooks2018NatCom}.
If one considers that dS projection neurons can bidirectionally regulate the output activity of the basal ganglia \cite{Kravitz2010}, our proposed function of the dS as computing action-specific effort signals can account for seemingly heterogeneous findings, such as modulations of the relative preference for a particular velocity \cite{ Yttri2016}, action \cite{Tai2012NN, Kravitz2012NN} or behavioral state \cite{Kravitz2010} by selective optogenetic stimulation of dS neurons.
Biologically, expending effort to produce faster movements allows limiting the temporal discounting of reward (i.e., cost of time, \cite{Shadmehr2019TINS}).
In sensory guided decision-making tasks, the cost of time can also be reduced by limiting the duration of deliberation \cite{Carland2019}. 
Interestingly, recent evidence supports a specific role of the basal ganglia in signaling the urgency to commit to a choice \cite{Thura2017Neuron,Carland2019}. 
Future studies should investigate whether signaling effort and urgency are the two sides of a unique function implemented in the basal ganglia to maximize the reward rate while minimizing costs.
