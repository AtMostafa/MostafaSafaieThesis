\section{Striatal Function}
\label{ch:disscusion:lesion}

The striatum can powerfully influence the production of purposive movements.
Indeed, it is well-known that striatal dysfunction is the primary cause of the motor impairments (akinesia, bradykinesia, levodopa-induced dyskinesia) seen in \gls{pd}~\cite{Mink1996,Hornykiewicz2006,McGregor2019Neuron}.
\Gls{da} depletion in \gls{pd} differentially affects striatal \gls{d1} and \gls{d2} expressing neurons.
Activation of striatal \gls{d1} (\textit{\gls{d2}}) \glspl{msn} forming the direct (\textit{indirect}) \gls{bg} pathway facilitates (\textit{prevents}) movement production through disinhibition (\textit{inhibition}) of brainstem and forebrain motor regions~\cite{Kravitz2010Nature}.
This fundamental feature of \gls{bg}'s functional anatomy, combined with recordings and perturbations of striatal activity in various behavioral tasks has led to two prevailing hypotheses regarding how the striatum contributes to the control of purposive movements: %#TODO: define purposive movement in a footnote
\begin{itemize}[noitemsep]
    \item learning and performing action sequences\footnotemark~\cite{Barnes2005Nature, Jin2010N, Cui2013Nature, Klaus2017Neuron, Markowitz2018Cell};
    \item  modulation of movement speed~\cite{Kim2014EJN, Rueda2015NN, Barbera2016Neuron, Yttri2016Nature, Panigrahi2015Cell}.
\end{itemize}
\footnotetext{
    Action sequences are sometimes referred to as ``procedural memories''~\cite{Barnes2005Nature}.
}
The validity of these hypotheses is debated~\cite[for instance, see][]{Dudman2016CurrOpinNeurobiol} and, interestingly, they face common shortcomings.
For example, they fall short of explaining why lesioning or inactivating striatum's anatomical targets (i.e., the \gls{bg} output nuclei) in non-human primates only marginally alters the execution and speed profile of overlearned motor sequences~\cite{Desmurget2010JNeurosci}; it even alleviates motor deficits observed in \gls{pd} patients~\cite{Turner2010CurrOpinNeurobiol}.
Moreover, in behavioral tasks typically used to probe the striatal motor function through perturbation of neuronal activity, it is next to impossible to disentangle whether failure to perform is due to inability to implement a decision into movement, or due to an impairment in higher-level processes (e.g., sensory processing and decision making), despite functional motor systems.\footnotemark
\footnotetext{
    This is referred to as the \emph{performance confound}.
    I only understood this concept after my supervisor came up with this rather bitter example:
    \textit{Imagine you cut up someone's legs and then conclude that they \underline{cannot} learn to run.}
}
\par
In this part of the study, to understand how the striatum contributes to the control of purposive actions, while limiting the performance confound as much as possible, we took advantage of the behavior displayed by animals in the treadmill task~(\autoref{fig:time:CtrlTrd}).
This task was identical to the one used in the time experiments under the `normal' condition (\Autoref{fig:methods:taskRules}{a}, and \Autoref{fig:lesion:task}{A}).
We trained a group of animals, including the same animals presented in \autoref{fig:time:CtrlTrd}.
They mostly developed the wait-and-run routine on the treadmill (but also see \autoref{fig:appendix:BadCtrl}).
After their performance on the task plateaued, they were randomly assigned to striatal lesion groups in different areas:
    \gls{dls}, \gls{dms}, or the entire \gls{ds}.
Excitotoxic lesions, the kind we used here (\autoref{ch:method:lesion}), compared to lesions induced by electrical currents, have the benefit of sparing the passing fibers.
Moreover, due to their permanent nature, lesions are probably a more direct way to assess the function of the manipulated area~\cite{Otchy2015Nature}.
Following the lesion, animals were allowed to recover for $\sim$2~weeks.
After this recovery period, visually, they had normal behavior in their homecage.
Then, they resumed the training with identical task parameters to the sessions prior to the lesion.
After the striatal lesion, most animals\footnotemark\ could still perform in the task, with comparable proficiency to that of the pre-lesion sessions.
\footnotetext{
    Except for two rats with the largest lesions ($>80\%$).
    These animals are, for instance, the two lines in \Autoref{fig:lesion:task}{H, right} that drop out of the axes limits.
}
In addition, motivation does not seem to be affected, since the animals still engaged in the task, and consumed the reward in correct (and thus rewarded) trials (\autoref{fig:lesion:lick}).
The most striking effect was a marked reduction in the running speed toward the reward.
This slowdown was irreversible (\Autoref{fig:lesion:task}{J}) and well-correlated with the size of the lesion (\Autoref{fig:appendix:spd}{A}).
The delayed consumption of the reward (\Autoref{fig:lesion:lick}{B}) is also another measure related to reduced speed, due to either slower locomotion toward the reward port, or slower postural adjustments to start licking.
Interestingly, animals with reduced speed, also started to run earlier toward the reward, i.e., the maximum position of their trajectories was smaller.
This trait was also well-correlated with the lesion size and was also irreversible (\autoref{fig:lesion:maxPos}), but not trivial to explain using the aforementioned models of the striatal function (more on this later).
\par
Noticeably, in a number of animals, arguably the ones with relatively larger lesions, executing the routine was impaired for the first few sessions ($\sim<$5), and recovered afterward.
They mostly stayed in front of the infrared beam and committed many error trials.
It could be argued that the lesion prevented the performance of the motor routine, and especially animals with larger lesions stayed in the front since it is associated with the reward, evident by the strong place preference for the reward area in both na\"{i}ve and trained rats (\autoref{fig:appendix:initPos}).
This is also supported by a recent preprint from \citeauthor{Dhawale2019} proposing that the \gls{dls}, and not \gls{dms}, controls the detailed structure of learned behaviors on a moment-to-moment basis, and thus \gls{dls} lesions bring back pre-learning behaviors~\cite{Dhawale2019}.
To further investigate this possibility, we designed a variant of the task to directly evaluate whether lesioned animals lost the ability to perform motor routines (or at least a comparable motor routine).
In this `reverse' treadmill task, the conveyor belt moved toward the reward area.
Animals could just move to the back of the treadmill during the relatively-long intertrial (while the treadmill motor was turned off, this would be a simple locomotion which was not affected in lesioned animals, \Autoref{fig:lesion:motorOk}{A}) and following trial start, stay still while the treadmill carried them to the reward area at the right time (\gls{et}~$\geq$~\gls{gt}).
Importantly, impaired execution of this ``run-and-wait'' motor routine could be clearly observed, because the animals should similarly stay in the front.
Thus, this task resolves the issue of the performance confound, since even a compromised motor system can execute this routine.
However, the experimental data confirmed, animals kept performing the routine with no sign of deficiency (\autoref{fig:lesion:rev}).
These results demonstrate the spared ability of animals to perform motor routines with lesioned striatum.
\par
We also formally confirmed normal locomotor activity by measuring the total displacement during the first 10~min in an unfamiliar environment (\Autoref{fig:lesion:motorOk}{A}, see \autoref{ch:methods:loco} for details).
Furthermore, we investigated whether lesioned animals were able to run at faster speeds.
Some of the control and lesioned animals were tested in a new paradigm, consisting in trials of 30~s followed by intertrials of 30~s.
There was no reward available and the speed of the treadmill progressively increased across trials (see \autoref{ch:methods:loco} for details).
Interestingly, lesioned animals were capable of running at speeds much higher than the treadmill speed---up to 40~cm/s compared to the treadmill speed of 10~cm/s (\Autoref{fig:lesion:motorOk}{B}).
Similar results have been reported in \gls{pd} patients using different behavioral paradigms~\cite{Mazzoni2007, Schmidt2008Brain}.
\par
Another possible explanation of our data may be lack of more general motor control abilities, such as modulation of speed.
For example, it has been previously shown that the speed of reward-oriented movements increases with movement distance to minimize the \gls{cot}~\cite{Shadmehr2010Jneurosci}.
In our task, one could argue that an animal with a limited range of locomotion speed at its disposal, would prefer a shorter running epoch with a constant slow pace, to a longer one with more complex speed dynamics (\Autoref{fig:lesion:motorOk}{C}).
By further analysis of our data, I found that, for every single control animal, trials with higher maximum position were indeed faster (\Autoref{fig:lesion:motorOk}{D}).
Importantly, this effect was preserved after striatal lesion, although all the speeds were generally lower than control animals.
These results show that animals' elementary ability to modulate their locomotion speed was also maintained after striatal lesion.
\par
Performing striatal lesions in na\"{i}ve animals and then training them with the usual protocol yields similar behaviors as control animals (\Autoref{fig:lesion:EarlyLesionLearning}{A}).
Learning the wait-and-run routine should also be added to the list of abilities preserved after the \gls{ds} lesion.
These ``early lesion'' animals are also slower, and their treadmill crossing speed anti-correlates with their lesion size (\Autoref{fig:appendix:spd}{B}).
Some animals with larger lesions,  however, failed to learn the task (\Autoref{fig:lesion:EarlyLesionLearning}{B}), they constantly stayed in the front of the treadmill.
This is reminiscent of the behavior of some rats, also with the largest lesions in \autoref{fig:lesion:task} and \autoref{fig:appendix:Frontal}.
Staying in the front could be considered as the extreme (and persistent) version of the `acute' effect observed after the lesion.
While after striatal lesion, majority of animals kept arriving on time in the reward area, but they started to run earlier (i.e., from a more frontal portion of the treadmill) and at a slower velocity (\autoref{fig:lesion:maxPos}, and \autoref{fig:appendix:AllRatMaxPos}).\footnotemark
\footnotetext{
    Concerning the timing mechanisms discussed in \autoref{ch:disscusion:time}, the temporal accuracy (standard deviation of \gls{et}) was also irreversibly increased following striatal lesions (\autoref{fig:appendix:SdofET}).
    This is consistent with the embodied argument that considers an important role for a salient sensory cue, such as touching the back wall of the treadmill in time estimation.
    Following the striatal lesion, increased variability might be due to the shorter waiting bout and starting to run before arriving to the back wall, which similar to the short~GT experiments (\autoref{fig:time:shortSharp}) led to increased variability.
    }
Running at slower speeds could be explained by current models of striatal function, but the tendency to start running early is surprising.
Alternatively, animals could have reached similar positions on the treadmill and still used a slower speed to approach the reward.
The major disadvantage of such hypothetical strategy may be a longer running distance which incurs more energetic cost.
Indeed, optimal control models predicted that higher sensitivity to energy expenditure (effort) leads to a similar strategy: starting to run earlier and slower~\cite{JuradoParras2020}.
Our results thus support the view that the striatal lesion increased animals' sensitivity to effort which led them to modify the kinematics of the wait-and-run routine.
In other words, our work suggests that \gls{ds} contributes to the generation of an effort signal that influences the kinematic parameters of purposive actions.
Therefore, the tendency to remain in the front of the treadmill after large \gls{ds} lesions in any of our experiments could simply reflect a heightened conservative energy policy.
Metaphorically speaking, the same effect would have been expected had we forced control non-lesioned rats to perform the task with extra weight on their back.
\par
Such a function is congruent with the hypothesis, derived from \gls{pd} patients, that \gls{da} projections to the striatum provide a signal for implicit motor motivation (or global effort sensitivity), which in turn influences the vigor of goal-directed movements~\cite{Mazzoni2007}, and lack thereof causes bradykinesia (also suggested by others like~\cite{Berke2018NN, Beeler2012Fron}).
In addition, progressive degeneration of \gls{da}ergic neurons in a mouse model of \gls{pd} suggests a critical role of \gls{da} in the \gls{ds} for the control of movement vigor~\cite{Panigrahi2015Cell}.
On the other hand, whether the \gls{ds} is critical for action selection/initiation/repression has been an important topic of debate~\cite{Turner2010CurrOpinNeurobiol, Dudman2016CurrOpinNeurobiol}.
In this context, our study provides compelling evidence for a specific role of the \gls{ds} in setting the sensitivity to effort.
Our results indicate that striatal lesions change the kinematics of a well-learned motor routine as a result of increased sensitivity to effort, without altering the animals' capacity to run at different speeds.
It is known that selective perturbation of the activity of striatal projection neurons bidirectionally modulates the speed of goal-directed movements~\cite{Yttri2016Nature} and spontaneous locomotion~\cite{Kravitz2010Nature}.
Our results complement these studies by suggesting that the \gls{ds} is not the primary controller of movements, but provides a second layer of modulation that tunes their kinematics according to cost/benefit considerations~\cite{Berke2018NN,Beeler2012Fron}.
More generally, the role of the \gls{bg} in contributing to the cost/benefit analysis of actions has a explanatory power beyond the kinematic modulation as it can reconcile different hypotheses regarding the \gls{bg} function.
Of course, I do not claim it can account for every single \gls{bg} study,\footnotemark\ however following, I present an alternative reading of several examples:
\footnotetext{
    For instance, our proposed function of the dorsal \gls{bg} to set the sensitivity to effort has implications for time estimation as well.
    \Citeauthor{Paton2016Sci} demonstrated that ``activation or inhibition of dopamine neurons was sufficient to slow down or speed up time estimation, respectively''~\cite{Paton2016Sci}.
    Assuming a prokinetic role for \gls{da}, our framework (embodied timing, plus the effort function of the \gls{bg}), would suggest that mice might have executed a subtle motor routine during the to-be-estimated interval as a reference.
    The execution of this routine presumably was sped up upon \gls{da} photo-stimulation, and slowed down by \gls{da} inhibition, based on the \autoref{ch:disscusion:lesion:p1}, causing overestimation and underestimation of the interval, respectively.
    This prediction, however in line with some previous reports~\cite[see][for a review]{Buhusi2005NatRevNeuro}, contradicts the results!
    One possible reason why might be that it is non-trivial how manipulating the activity of \gls{da} neurons is reflected in actual \gls{da} release in the striatum~\cite[for instance, see][]{Mohebi2019N}.
}
\begin{enumerate}[noitemsep, label=\Roman*.]
    \item \label{ch:disscusion:lesion:p1} The seminal work of \citeauthor{Kravitz2010Nature}~\cite{Kravitz2010Nature} could be reinterpreted as cost overestimation caused by indirect pathway activation that manifests as freezing, and cost underestimation by direct pathway activation that promotes locomotion.
    Therefore, this framework also predicts a prokinetic role for \gls{da} in the striatum~\cite{Howe2016N}, due to its dual effect on \glspl{msn}.
    \item Striatal \glspl{msn} are reported to be most active at the start/stop of a stereotyped motor sequence~\cite{Barnes2005Nature, Jin2014NN}, which is also the most intense time to compute the effort signal from all the incoming sensorimotor information, since the behavioral state is drastically changing~\cite{Sales-Carbonell2018, Robbe2018}.
    \item The insightful paper of \Citeauthor{Yttri2016Nature}~\cite{Yttri2016Nature} shows that activation of either direct and indirect pathway is sufficient to produce sustained increase and decrease in movement velocity.
    Similar results are predicted with the cost hypothesis, assuming the bidirectional control of cost estimation by direct and indirect pathways, the same as the \autoref{ch:disscusion:lesion:p1} above:
    actions deemed less costly due to the stimulation of \gls{d1} \glspl{msn} are later preferred, and those co-occurring with \gls{d2} stimulation are less likely to be repeated.
    \item Many works in the instrumental learning field also reported altered engagement in the task (usually different lever press rate) as a sign of impaired learning and/or execution~\cite{Hart2018CurrBiol}.
    In a recent example~\cite{Matamales2020Sci}, in two similar tasks, mice with genetic ablation of \gls{d2} expressing neurons showed increased lever-press rates compared to the control mice.
    This, once again, is in line with our hypothesis, i.e., lever pressing in the absence of the `skeptic' \gls{d2} neurons is more likely.\footnote{
        ``The direct–indirect competition implements a decision by weighing the arguments of a Believer (e.g., direct pathway) against those of a Skeptic (e.g., indirect pathway).
        Because the default state of the BG is heavily motor suppressing\ldots, the burden of proof falls on the Believer and thus actions are only executed when the accrued evidence sufficiently reduces the Skeptic’s uncertainty''~\cite{Dunovan2016FrontNeurosci}.
    }
\end{enumerate}
\par
Expending effort to produce faster movements lowers the \gls{cot} as well~\cite{Shadmehr2019TINS}.
In sensory guided decision-making tasks, the \gls{cot} can also be reduced by limiting the duration of deliberation~\cite{Carland2019NeuroSci}.
Interestingly, recent evidence supports a specific role of the \gls{bg} in signaling the urgency to commit to an action choice~\cite{Thura2017Neruon,Carland2019NeuroSci}.
Thus, our proposed function of the \gls{ds} might provide a common framework to reconcile seemingly heterogenous findings across motor control and decision-making fields.