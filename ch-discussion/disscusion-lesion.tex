\section{Striatal Function}
\label{ch:disscusion:lesion}
Our results support the view that the striatal lesion increased the animals' sensitivity to effort which led them to modify the kinematics of the wait-and-run routine.
Theoretically, remaining very close to the reward area minimizes energy expenditure (effort) by avoiding the usage of fast speeds.
This extreme strategy was observed during the first post-lesion session of a few rats with large lesions (Fig. 1F and G).
But it exposed these animals to premature entrances in the reward area (Fig. 1H) and consequently, to an abrupt reduction in rewards obtained during this session.
Across post-lesion sessions, the same rats progressively waited longer before running toward the reward area, which allowed them to recover task proficiency (Fig. 1, C to F).
This suggested that the animals' ability to progressively wait longer before running toward the reward area was spared by the dS lesion.
Accordingly, we found that dS lesions performed before training did not compromise the rats' ability to learn the wait-and-run routine (fig. S7) but did reduce their speed, an effect that was also correlated with lesion size (fig. S4). 
\par
Overall, our results indicate that rats with dS lesions did not display fundamental impairment in motor control or action selection but behaved in a way that is most parsimoniously  explained by a higher sensitivity to effort with preserved motivation.
Indeed, after dS lesion, animals kept arriving on time in the reward area, but they started to run earlier (i.e., on a more intermediate portion of the treadmill) and at a slower speed.
Metaphorically speaking, the same effect would have been expected had we forced non-lesioned rats to perform the task with extra weight on their back. 
Thus, our work suggests that the dS contributes to the generation of an effort signal that influences the kinematic parameters of purposive actions.
Such a function is in line with the hypothesis that the dopamine projection to the dS provides a signal for implicit motor motivation (or global effort sensitivity), which in turn influences the vigor of most of the goal-directed movements performed human or animal subjects \cite{Mazzoni2007JN, Treadway2012JN, Reppert2018JNPhys}.
But, what would be the specific contribution of dS neurons to such function?
A possible answer is that dS neurons integrate motivational- and feedback-related dopaminergic signals with context- and action-specific information derived from their massive cortical and thalamic inputs  \cite{Hunnicutt2016Elife, Hooks2018NatCom}.
If one considers that dS projection neurons can bidirectionally regulate the output activity of the basal ganglia \cite{Kravitz2010}, our proposed function of the dS as computing action-specific effort signals can account for seemingly heterogeneous findings, such as modulations of the relative preference for a particular velocity \cite{ Yttri2016}, action \cite{Tai2012NN, Kravitz2012NN} or behavioral state \cite{Kravitz2010} by selective optogenetic stimulation of dS neurons.
Biologically, expending effort to produce faster movements allows limiting the temporal discounting of reward (i.e., cost of time, \cite{Shadmehr2019TINS}).
In sensory guided decision-making tasks, the cost of time can also be reduced by limiting the duration of deliberation \cite{Carland2019}. 
Interestingly, recent evidence supports a specific role of the basal ganglia in signaling the urgency to commit to a choice \cite{Thura2017Neuron,Carland2019}. 
Future studies should investigate whether signaling effort and urgency are the two sides of a unique function implemented in the basal ganglia to maximize the reward rate while minimizing costs.
