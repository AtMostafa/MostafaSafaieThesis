\section{Time Estimation}
\label{ch:disscusion:time}

In this study, we used a treadmill-based behavioral assay in which rats, once the treadmill started, were required to wait for 7~s before approaching the reward location.
Objectively, animals may accurately time their approaches using either one of the following two mechanisms.
First, they may rely on a purely \textit{internal} mechanism (e.g., self-sustained neuronal dynamics read by their motor system) to learn how long they should wait and decide when to approach the reward port.
In that case, performance accuracy should be largely independent of variations in \textit{external} factors (e.g., the speed of the treadmill, the animals position on the treadmill at trial onset,\ldots).
In addition, animals would probably stay close to the reward area for most of the duration of the trial.
Alternatively, animals may discover by trial-and-error a motor routine adapted to the apparatus and task parameters whose complete execution would take them into the reward area at the right time, i.e., the \gls{gt}.
In that case, timing accuracy would be related to the stereotyped performance of that routine and should heavily depend on task-specific features of the environment or the order of the elements composing the motor sequence.
The dominance of either of the algorithms can be directly inferred from behavioral experiments in which critical task parameters are manipulated.
The results of our behavioral experiments clearly favor the latter embodied strategy.
Using two distinct reinforcement learning-based algorithms that either incorporated or lacked time representation, we showed that the behavior of our animals is incongruent with them accessing an internal explicit knowledge of elapsed time~\cite{Safaie2020PNAS}.
\par
We report that to accurately wait 7~seconds before approaching the reward port, most rats developed the following ``wait-and-run'' motor routine.
First, they waited for the beginning of each trial in the reward area.
Then, upon trial onset, they stayed relatively still while the treadmill carried them to the rear wall of the treadmill.
Finally, as soon as they reached the back of the treadmill, they ran straight to the reward port, without pause.
In this experimental ``control'' condition (see \autoref{ch:time:treadmill}), the accuracy of the animals reached its peak after 15~to 20~training sessions.
However, even for proficient animals, the probability of performing a correct trial was almost null when they started a trial in the back region of the treadmill.
In addition, when animals started a trial in the reward area, performing a correct trial was almost exclusively associated with the animals reaching the back portion of the treadmill.
Finally, following extensive training in the control condition, when we modified the task parameters to penalize the stereotyped performance of the front-back-front trajectory, the behavioral proficiency and accuracy of the animals dropped dramatically.
These results support the hypothesis that, in our task, performing the motor routine is necessary for accurate performance.
\par
It could be argued that the animals' tendency to develop this front-back-front trajectory resulted from the structure of the task that provided an easy solution that animals used instead of estimating time while continuously running just behind the infrared beam.
In other words, in conditions that do not favor the usage of a simple motor routine, rats may time their reward approaches by relying on an internal representation of time that arises from the ability of recurrent neural networks to generate self-sustained time-varying patterns of neural activity\cite{BuonomanoTICS2010}.
However, several additional experiments confirmed that rats have limited ability to use an internal representation of time when the task parameters are set such as to prevent the usage of a stereotyped motor sequence to solve the task.
First, we trained a group of animals while the treadmill speed randomly changed across trials.
Compared to animals trained in the control condition, those trained with variable speed were less accurate.
Additionally, these animals attempted to use the front-back-front trajectory as shown by an increased probability of correct trials when the treadmill speed allowed it.
Second, we trained a different group of rats in a version of the task that penalized them when they started the trials in the reward area.
In this condition, solving the task is not possible using the wait-and-run routine since they would generate early entrance times in the reward area.
Rats trained in this condition displayed strong accuracy impairment.
Moreover, they kept trying to develop a modified front-back-front trajectory and started the trials as close as possible to the infrared beam.
In all the above experiments, during trials, the treadmill pushed the animals away from the reward area which favors the usage of the wait-and-run routine.
To avoid this possible bias, in our last experiment, we trained a group of rats on an immobile treadmill.
Rats' performance was poor in this condition, with some animals failing to show any signs of learning.
Furthermore, animals that did eventually learn, performed a modified motor sequence during which they ran to the back, performed some movements in the back of the treadmill (that we could not quantify with our video tracking system) and then rushed back to the front.
Altogether, we conclude from this set of experiments that rats, forced to wait for several seconds before doing a certain action, did not seem capable of using an internal disembodied representation of time, but always attempted to develop a motor routine in the confined space of the treadmill, routine whose execution duration amounted to the time they needed to wait.
\par
This conclusion was also supported by the fact that animals were less accurate in timing their entrance in the reward area when the goal time was set to 3.5~s, compared to the control goal time (7~s).
Indeed, in this short goal time condition, the wait-and-run strategy is not optimal, as animals would enter the reward area too late.
Thus, the increased variability might be explained by the difficulty for the rats to "self-estimate" when to start running forward without the help of a clear sensory cue (such as touching the back wall).
In support of this idea, in 67\% of the error trials, the rats started running forward before reaching the middle of the treadmill.
In addition, a few animals trained in the short goal time condition developed a new stereotyped motor sequence (running from front to back and back to front).
Interestingly, their entrance times were less variable than animals that remained immobile after trial onset and tried to estimate when to run forward in the middle portion of the treadmill.
We noticed that the best rats in the immobile treadmill condition systematically ran to the back region of the treadmill where they performed a series of rearing and wall-touching movements, just before crossing the treadmill toward the reward area.
Thus, the increased variability is likely to result from the fact that, when the treadmill is immobile, a motor sequence to fit in 7~s is more difficult to reproduce reliably than in the control condition in which most of the sequence is a passive wait on the treadmill until the animal reached the rear wall.
Thus, the scalar property of variance may primarily reflect a difficulty in performing long motor sequences, rather than long time itself.
A more practical limitation of our work is whether its conclusion is relevant beyond the specifics of our experimental protocols (a supra-second long motor timing task in which the rewarding action is a full-body movement in space).
Interestingly, in a study in which a group of rats had to perform two lever presses interleaved by 700~ms, each animal developed an idiosyncratic motor sequence (e.g., 1\# first press on the lever with the left paw; 2\# touching the wall above the lever with the right paw; 3\# second press on the lever with the left paw), lasting precisely 700~ms\cite{Kawai2015Neuron}.
The large inter-individual variability reported in this study may arise from the multiple possibilities of simple action sequences that can be squeezed in such a short time interval and easily reproduced across trials, taking advantage of the proximity of the front wall and lever.
If the time interval was longer, all the animals might have developed the same motor sequence (e.g., running back and forth in the experimental cage between the two lever presses).
Nevertheless, this study provides an additional example in which virtually all animals developed a motor strategy, even if, compared to our task, the time interval was much shorter ($< 1$~s) and the terminal operant response was distinct (a single lever press).
\par
It is well known that temporal regularities in animal conditioning protocols favor the development of automatic motor sequences.
More remarkably, in one of the rare studies that continuously recorded and quantified the full body dynamics of rats performing a sensory duration categorization choice task, it was reported that animals developed highly stereotyped motor sequences during presentation of the sensory cues and that perceptual report of the animals could be predicted by these motor sequences \cite{Gouvea2014FrontNeurorobotics}.
Importantly, the choice of the animal could be predicted by examining the execution of the motor sequence.
Thus, rats may also use an embodied strategy in tasks requiring them to categorize (short or long) the duration of time intervals.
This result is reminiscent of an earlier study showing that the prediction of rats' temporal judgement (a 6~s long versus a 12~s long luminous signal) was always better if based on the collateral behavior performed by the animal at the end of the signal than if based on time\cite{Fetterman1998}.
Thus, in such temporal discrimination tasks, a stereotyped sequences of movements (collateral behavior) might serve as an external clock and the choice of the animals might be primarily determined by what the animal is doing when a sensory cue disappears rather than by an internal estimation of the duration of that cue.
Altogether, these studies support the idea that animals resort to motor strategies to adapt to temporal constraints in a wide range of timing tasks.
The novelty of our work is, first, to demonstrate that even in conditions that discourage the use of such motor strategies rats do not seem able to rely on a purely internal timing mechanism and, second, that a critical determinant of temporal accuracy is the possibility to develop motor routines that can be guided by interactions with salient features of the environment.
\par
Our conclusion that animals rely on an embodied strategy, rather than internal neuronal clocks (dedicated or emergent), does not mean that internal brain activity is irrelevant to time-related behavior.
We do not question that representations of elapsed time have been observed in individual and population neuronal activity in various brain regions during time-constrained tasks or that perturbation of neuronal activity impairs timing accuracy or discrimination.
It has been previously proposed that timing could be mediated through motor routines whose precise execution is internally controlled\cite{Killeen1988,Dragoi2003, Staddon1999,Machado1997}.
Thus, it could be argued that accurate timing in our task was ultimately driven by internal neuronal dynamics.
We don't dispute the fact that neuronal activity is required for proficient performance in our task.
Actually, we have previously reported that striatal inactivation decreased timing accuracy in a slightly modified version of this task\cite{Rueda2015NatNeuro}.
In addition, there is no reason why the moment-to-moment movement dynamics of the animals on the treadmill could not be decoded from spiking activities recorded across cortical and subcortical regions.
However, this type of result can not be used as a definitive evidence in favor of a neuronal representation of time read by the animals as we, humans, watch a clock\cite{Krakauer2017Neuron, Buzsaki2017Science, Buzsaki2018TICS}.
Our behavioral results are not easily compatible with the idea that neural representations of time are a signature of a clock-like algorithm for time estimation.
Indeed, here we report that timing accuracy was reduced when the task parameters prevented the animals from taking advantage of the physical structure of the treadmill to learn the motor routine.
Thus, in our task, something more than an internal process (be it a dedicated clock or the self-sustained population dynamics emerging from recurrently connected circuits)  was required for accurate timing:
the reciprocal and repetitive interactions between the nervous system and the body (sensors and actuators) on the one hand, and the surrounding environment on the other hand.
Our results are compatible with the idea that timing emerges from the dynamics of neural circuits \cite{Paton2018Neuron,Goel2014PhilTrans}, as long as these dynamics are not entirely internally generated but also reflect feedback from the environment.
For instance, we would assume that the timing deficits induced by striatal inactivation \cite{Rueda2015NatNeuro} might be explained by the role of this brain region in accumulating sensory information before taking a decision \cite{Yartsev2018eLife,Dunovan2016FrontNeuro}.
par
That timing could be primarily embodied and situated might seem counterintuitive with our innerly rooted feeling of time.
Nevertheless, it is interesting to note that to precisely measure time, we have created devices that indicate time by moving objects in space and extensively use metaphors containing movement and space references when speaking of time ("holidays are approaching", "time flies")\cite{Nunez2013TICS,Winter2015Cortex}.
Moreover, humans display poor temporal judgment accuracy when prevented to count covertly or overtly\cite{Rattat2012} and several studies have reported that movements improve the perception of rhythmical intervals\cite{Su2012,Manning2013,Wiener2019eNeuro}
It has been recently proposed that the explicit perception of time in humans may be constructed implicitly through the association between the duration of an interval and its sensorimotor content\cite{Coull2018TICS}.
The fact that motor timing may be fundamentally related to movement in space for both animals and humans could explain why brain regions involved in movement control and spatial representation, such as the motor cortex, basal ganglia, cerebellum and hippocampus, have consistently been associated with time representation\cite{pouthas2005HumBrainMapp, Kraus2013Neuron, Bakhurin2017JNeurosci, Morillon2017PNAS, Gu2018NeurobLearnMem, Mello2015CurBio, Villette2015, Pastalkova2008, Schubotz2000}.
Still, why animals and humans seem to favor embodied and interactive timing strategies over purely internal mechanisms is not clear.
Insights regarding this question might be obtained by considering adaptive behavior in an evolutionary perspective\cite{Cisek2019} and time in the context of ecologically valid timing tasks\cite{vanRijn2018}.
In addition, we found that an artificial agent endowed with time representation solved a virtual implementation of our task in a different way from rats.
Nevertheless, the embodied view of time perception is compatible with the idea that timing is associated with spiking dynamics of neuronal populations distributed across several brain regions, including motor-related cortical and subcortical regions\cite{Paton2018Neuron,Goel2014PhilTrans}.
Our work specifically suggests that this population activity (neuronal ensemble trajectories) is constrained by the dynamics of movements and sensory signals.
Thus, well-timed behavior emerges from reciprocal and repetitive interactions between the nervous system, the body (sensors and actuators) and the surrounding environment. 
\par
In our experimental setting, rats may decide when to run forward after gathering sufficient evidence (based on a mixture of visual information, sensory stimulation of their tail or their back).
Thus, it may be relevant to consider the process governing when the rats will run forward as an accumulation of sensorimotor evidence.
The dorsal striatum is critical for processing sensorimotor information\cite{Robbe2018CON} and has been proposed to contribute to the process of evidence accumulation during decision making\cite{Yartsev2018eLife}.
Interestingly, it has been recently proposed that a competition between the direct and indirect basal ganglia pathways, tuned by dopaminergic modulation, may determine the speed of evidence accumulation toward decision taking\cite{Dunovan2016FrontNeuro,Dunovan2019JNeurosci}.
Such a model predicts that, in the context of our task, an increase (decrease) in dopaminergic activity will speed up (slow down) the accumulation of sensorimotor information and will lead to an early (delayed) response.
Such a scenario is in agreement with a recent study in mice performing an auditory duration discrimination task, showing that increased dopaminergic activity in the substantia nigra compacta was associated with the animals perceiving long tones as short ones\cite{Soares2016Science}.
Future work should further investigate how to mathematically capture the algorithms underlying temporally constrained actions and decisions and the possible contribution of the striatum to processing multimodal (sensorimotor, cognitive, emotional) contextual information together with dopamine-mediated feedback and motivational signals.