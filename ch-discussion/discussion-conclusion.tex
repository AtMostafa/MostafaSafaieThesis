\section{Conclusion} \label{ch:discussion:conclusion}

In short, accurate timing requires stereotyped interaction with the environment and the striatum determines the effort invested in this interaction~\cite{Safaie2020PNAS,JuradoParras2020}.
\par
That perception of elapsed time is related to the movement of things is not a surprise to anyone.
In this work, however, we attempted to determine the necessity of movements.
In other words, whether there is an internal mechanism which can provide a measure of time that drives behavior, or time is perceived through actions that fill the interval.
Multiple experiments designed to interfere with the usage of stereotyped motor routines, all led to drastic decline in temporal accuracy.
These results suggest that the hypothetical internal timer does not suffice for producing a timely motor response in the suprasecond time scale.
We thus argue that the representation of time intervals in the brain may be replaced with movements that happen to take that long to execute.
In turn, temporal control of individual sub-actions in a complex movement presents a different timing problem, which is in a much shorter timescale (tens of milliseconds, instead of seconds), and constrained with the mechanical characteristics of the body (e.g., mass, leg length,\dots).
Thus, the problem of time perception is translated to a motor control problem, learning and performing adaptive motor routines.
\par
The \glsentrylong{ds} has been implicated in initiation and selection of movements, as well as in controlling their speed.
We took advantage of the wait-and-run motor routine, consistently developed by animals in the treadmill task, to delineate the role of the striatum.
For a large group of animals, permanently lesioning the dorsal striatum did not affect the ability to learn the motor routine by trial and error, to perform the learned motor routine, to run fast, nor to modulate the running speed as needed.
However, lesioned animals robustly performed a less effortful routine by waiting less and running slower.
Hence, we propose that the main motor function of the stratum may be setting the sensitivity to effort in purposive actions.
Such a low-level function can potentially reconcile a large body of seemingly contrasting hypotheses.