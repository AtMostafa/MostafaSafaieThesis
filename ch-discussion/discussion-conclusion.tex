\section{Conclusion} \label{ch:discussion:conclusion}

In short, accurate timing requires stereotyped interaction with the environment and the striatum determines the effort invested in this interaction~\cite{Safaie2020PNAS,JuradoParras2020}.
\par
That the perception of elapsed time is somehow related to the movement of things is not a surprise to anyone.
In this work, however, we attempted to determine the necessity of movements.
In other words, whether there is an internal mechanism which can provide a measure of time that drives behavior, or time is perceived through actions that fill the interval of interest.
Multiple experiments designed to interfere with the usage of stereotyped motor routines, all led to drastic decline in temporal accuracy.
These results suggest that the hypothetical internal timer does not suffice to produce a timely motor response in the suprasecond timescale.
We thus argue that the representation of time intervals in the brain may be replaced with movements that happen to take that long to execute.
% In turn, temporal control of individual sub-actions in a complex movement presents a different timing problem, which is not only in a much shorter timescale (tens of milliseconds, instead of seconds), but also it's constrained with the mechanical characteristics of the body (e.g., mass, leg length,\dots).
Thus, the problem of time perception is translated to a motor control problem, learning and performing adaptive motor routines.
\par
The \glsentrylong{ds} has been implicated in initiation and selection of movements, as well as in controlling their speed.
We took advantage of the wait-and-run motor routine, consistently developed by animals in the treadmill task, to delineate the role of the striatum.
Using a number of original tasks, for a large group of animals, we illustrated that permanently lesioning different areas of the dorsal striatum does not affect neither the ability to learn the motor routine by trial and error, to perform the learned motor routine, to run fast, to modulate the running speed as needed, nor the motivation to acquire the reward.
However, lesioned animals robustly performed a less effortful routine by waiting less and running slower, two traits that well-correlated with the size of the lesion.
Hence, we propose that the main motor function of the striatum may be setting the sensitivity to effort in purposive actions.
Such an elementary function has the potential to reconcile a large body of seemingly contrasting hypotheses.